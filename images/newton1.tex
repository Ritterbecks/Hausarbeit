%% Autor: Björn Ritterbecks 
%% Letzte Aenderung: 15.06.2016 
\begin{marginfigure}
\centering
%\sansmath
\begin{tikzpicture}[
	scale=1,
	ka roehre/.style={fill=white,draw=black!80}
]
\begin{scope}[scale=1.5, rotate=90]
%Wände

\draw (-2,1) -- ++ (4, 0);
\shade[right color=halfgray!50, left color=halfgray!10] (-2, 1) rectangle (2, 1.3);
\shade[bottom color=mycolor!25, top color=mycolor!50] (1.5,1) rectangle (-1.5,-1);
\draw (-2,-1) -- ++ (4, 0);
\shade[left color=halfgray!75, right color=halfgray!25] (-2, -1) rectangle (2, -1.3);
%Höhe
 \draw[<->,>={latex[length=0pt 3*4,width=0pt 4]},thick, every node/.style={fill=white,midway}]   (1.8,1) -- ++ (0, -2.0) node {\footnotesize $h$};
   \draw[<->,>={latex[length=0pt 3*4,width=0pt 4]},thick, every node/.style={fill=none,midway}]   (0.375, -0.40) -- ++ (-0.75,0) node [right=0.03] {\footnotesize $\Delta \dot{x}$};
% Delta h
\draw[dashed, thin] (-1.5, 0.25) -- ++ (-0.2, 0);
\draw[dashed, thin] (-1.5, -0.25) -- ++ (-0.2, 0);
\draw[thin] (-1.5, 1) -- ++ (0, -2);
% Delta \dot{x}
\draw[dashed, thin] (0.375, 0.25) -- ++ (0, -0.75);
\draw[dashed, thin] (-0.375, -0.25) -- ++ (0, -0.25);
\draw [<->,>={latex[length=0pt 3*4,width=0pt 4]},thick, every node/.style={fill=white,midway}] (-1.6,0.25) -- ++ (-0,-0.5) node [below=0.05] {\footnotesize $\Delta h$}; 
  \node at (-0.1,1.14) {\footnotesize$\dot{x}_1$};   
% Geschwindigkeiten
\foreach \x in {0,...,7}
        {
            \draw[->,>={Triangle[length=0pt 3*4,width=0pt 4]}, mycolor4!50!mycolor2, thick] ({-1.5,1-\x*0.25}) -- ++ ({3-\x*0.375},0);
        };  
\draw[dashed] (1.5, 1) -- ++ (-3, -2);

\end{scope}
\end{tikzpicture}
  \caption[Herleitung des \textsc{Newton}'schen Reibungsgesetzes]{Herleitung des \textsc{Newton}'schen Reibungsgesetzes: Eine Flüssigkeit ist zwischen zwei Platten eingebettet. Die linke Platte bewegt sich mit der Geschwindigkeit $\dot{x}_1$ und zieht durch die \textsw{Adhäsionskraft} die Flüssgkeit an der Grenzfläche mit, die wiederum benachtbarte Flüssigkeitsschichten mit verringerter Geschwindigkeit $\dot{x}_2$ in Bewegung versetzt (nach \cite[S\,128]{Stroppe2008}).}
  \label{fig:newton1}
  \vspace{-0pt}
\end{marginfigure}