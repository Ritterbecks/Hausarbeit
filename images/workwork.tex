  \thisfloatsetup{%
  capbesidewidth=\marginparwidth
  }
\begin{figure}[htbp]
\centering
%\sansmath
\subfloat[]{
\begin{tikzpicture}
\begin{scope}[scale=0.98]
\begin{axis}[	
	clip=false, % Verhindet weiße Punkte bei vielfach geplotteten x-Werten
	ymajorgrids,
    xmajorgrids,
          axis x line*=middle,
          axis y line*=center,
    grid style={white},
	axis on top,
    width=\textwidth,  
    height=6.797cm,
    xmin=0,
    xmax=30,
    ymin=0,
    ymax=1200,
    %/tikz/ybar interval,
    tick align=center,
    xtick align=outside,
    xlabel={Entfernung $s$},
    ylabel={\rotatebox[origin=b]{270}{Arbeit $W$}},   
   x label style={at={(axis cs:25,-150)}},
    y label style={at={(axis cs:-3.0,780)}},
    axis line style={Honeydew4!70!black},
    ticklabel style={Honeydew4!70!black, inner sep=1pt,
                font=\footnotesize},
    yticklabels={${200}$, ${400}$, ${600}$,  ${800}$,  ${\si{\milli\newton\centi\metre}}$, ${1200}$ },            
    ytick={200,400,600, 800, 1000, 1200},
    xticklabels={0, 5, 10, 15, 20, $\si{\centi\metre}$, 30},
     xtick={0,5,10,15,20,25,30}, 
    scaled ticks=false,           
           legend style ={ at={(axis cs:30,1200)}, 
     anchor=north east, draw=none, 
         fill=none,align=left, text=Honeydew4!70!black,
          font=\footnotesize},               
    label style={font=\small,Honeydew4!70!black,},
    cycle list name=mycolor4 white,
    enlarge x limits=true,
    %tick style={draw=none},
    x tick label as interval=false,
]
\draw[thick, Honeydew4!70!black, ->,>={latex[Honeydew4!70!black, round, length=0.4cm, width=4pt]}] (axis cs:-5.50, 850) -- (axis cs:-5.50, 1150);
\draw[thick,Honeydew4!70!black, ->,>={latex[Honeydew4!70!black,round, length=0.4cm, width=4pt]}] (axis cs:21.00, -150) -- (axis cs:29.00, -150);
\addplot
table[x = x, y =y]{images/data/work1.dat};
\addlegendentry{$\SI{17.5}{\milli\metre}$};
\addplot
table[x = x, y =y]{images/data/work2.dat};
\addlegendentry{$\SI{15.0}{\milli\metre}$};
\addplot
table[x = x, y =y]{images/data/work3.dat};
\addlegendentry{$\SI{12.5}{\milli\metre}$};
\addplot
table[x = x, y =y]{images/data/work4.dat};
\addlegendentry{$\SI{10.0}{\milli\metre}$};
\end{axis}
\end{scope}
\clipright
\end{tikzpicture}}
\\
\subfloat[]{
\begin{tikzpicture}
\begin{scope}[scale=0.98]
\begin{axis}[	
	clip=false, % Verhindet weiße Punkte bei vielfach geplotteten x-Werten
	ymajorgrids,
    xmajorgrids,
          axis x line*=middle,
          axis y line*=center,
    grid style={white},
%	axis on top,
    width=\textwidth,  
    height=6.797cm,
    xmin=0,
    xmax=30,
    ymin=0,
    ymax=1200,
    %/tikz/ybar interval,
    tick align=center,
    xtick align=outside,
    xlabel={Entfernung $s$},
    ylabel={\rotatebox[origin=b]{270}{Arbeit $W$}},   
   x label style={at={(axis cs:25,-150)}},
    y label style={at={(axis cs:-3.0,780)}},
    axis line style={Honeydew4!70!black},
    ticklabel style={Honeydew4!70!black, inner sep=1pt,
                font=\footnotesize},
    yticklabels={${200}$, ${400}$, ${600}$,  ${800}$,  ${\si{\milli\newton\centi\metre}}$, ${1200}$ },            
    ytick={200,400,600, 800, 1000, 1200},
    xticklabels={0, 5, 10, 15, 20, $\si{\centi\metre}$, 30},
     xtick={0,5,10,15,20,25,30}, 
    scaled ticks=false,           
           legend style ={ at={(axis cs:30,1200)}, 
     anchor=north east, draw=none, 
         fill=none,align=left, text=Honeydew4!70!black,
          font=\footnotesize},               
    label style={font=\small,Honeydew4!70!black,},
    cycle list name=mycolor white,
    enlarge x limits=true,
    %tick style={draw=none},
    x tick label as interval=false,
]
\draw[thick, Honeydew4!70!black, ->,>={latex[Honeydew4!70!black, round, length=0.4cm, width=4pt]}] (axis cs:-5.50, 850) -- (axis cs:-5.50, 1150);
\draw[thick,Honeydew4!70!black, ->,>={latex[Honeydew4!70!black,round, length=0.4cm, width=4pt]}] (axis cs:21.00, -150) -- (axis cs:29.00, -150);
\addplot
table[x = x, y =y]{images/data/work1.dat};
\addlegendentry{$\SI{17.5}{\milli\metre}$};
\addplot
table[x = x, y =y]{images/data/work5.dat};
\addlegendentry{$\SI{15.0}{\milli\metre}$};
\addplot
table[x = x, y =y]{images/data/work6.dat};
\addlegendentry{$\SI{12.5}{\milli\metre}$};
\addplot
table[x = x, y =y]{images/data/work7.dat};
\addlegendentry{$\SI{10.0}{\milli\metre}$};
\addplot[mycolor, fill=none, smooth, domain=0:31,thick] {0.7145*\x^2-45.18*\x+1093};
\end{axis}
\end{scope}
\clipright
\end{tikzpicture}}
  \caption[An einer vorbeirollenden Kugel verrichtete Arbeit]{Geplottet sind die Arbeitsintegrale nach Gleichung \eqref{eq:gebl2}. {\color{mycolor}\textbf{(a)}:} Je kleiner der Kugelradius $r$, desto geringer ist die Strömungswiderstandskraft und damit auch das Integral über selbige. {\color{mycolor}\textbf{(a)}:} Durch Multiplikation der Arbeitsintegrale mit den Proportionalitätsfaktoren, welche aus den Kugelprojektionsflächen berechnet werden, kann die lineare Skalierung gezeigt werden.}
  \label{fig:workwork}
  \vspace{-0pt}
\end{figure}