%% Autor: Björn Ritterbecks 
%% Letzte Aenderung: 15.06.2016 
\thisfloatsetup{%
  capbesidewidth=\marginparwidth}
\begin{figure}[htbp]
\centering
\usetikzlibrary{decorations.pathmorphing}
\pgfplotsset{width=7cm,compat=1.13}
\small
%\sansmath
\begin{tikzpicture}[even odd rule,
	scale=1,
	ka roehre/.style={fill=white,draw=black!80}
]
\begin{scope}[scale=1.43]
\begin{scope}[xshift=0.25cm, yshift= 0.25cm]
% Helmholtz-Spule hinten
\path[top color=halfgray!50, bottom color=halfgray!25, draw=black] (0, 0) circle[radius=2.5] circle[radius=2.2];
\end{scope}
%B-Feld
\foreach \x in {0,...,6}
        { \foreach \y in {0,...,4} {
            \draw ({-1.5+0.5*\x},{-1.0+0.5*\y}) node[mycolor2] {\textbf{.}};
            \draw[mycolor2] ({-1.5+0.5*\x},{-1.0+0.5*\y}) circle (3pt);
        };      
};
            \draw (0,-2) node[mycolor2] {\textbf{.}};
            \draw[mycolor2] (0,-2) circle (3pt);
            \draw (0,2) node[mycolor2] {\textbf{.}};
            \draw[mycolor2] (0,2) circle (3pt);
                        \draw (-2,0) node[mycolor2] {\textbf{.}};
                        \draw[mycolor2] (-2,0) circle (3pt);
                        \draw (2,0) node[mycolor2] {\textbf{.}};
                        \draw[mycolor2] (2,0) circle (3pt);
\foreach \x in {0,...,4}
        { 
            \draw ({-1.0+0.5*\x},-1.5) node[mycolor2] {\textbf{.}};
            \draw[mycolor2] ({-1.0+0.5*\x},-1.5) circle (3pt);
        };
\foreach \x in {0,...,4}
        { 
            \draw ({-1.0+0.5*\x},1.5) node[mycolor2] {\textbf{.}};
            \draw[mycolor2] ({-1.0+0.5*\x},1.5) circle (3pt);
        };                              
% Neon-Dampf
\foreach \x in {1,...,20}
        {
            \pgfmathrandominteger{\a}{0}{359}
            \pgfmathrandominteger{\b}{175}{195}
            \shade[ball color=mycolor5!50, opacity=1] ({0.01*\b*cos(\a)}, {0.01*\b*sin(\a)}) circle (1.5pt);
        }; 
\foreach \x in {1,...,50}
        {
            \pgfmathrandominteger{\a}{0}{359}
            \pgfmathrandominteger{\b}{20}{160}
            \shade[ball color=mycolor5!50, opacity=1] ({0.01*\b*cos(\a)}, {0.01*\b*sin(\a)}) circle (1.5pt);
        };      
\foreach \x in {1,...,10}
        {
            \pgfmathrandominteger{\a}{220}{480}
            \pgfmathrandominteger{\b}{165}{173}
            \shade[ball color=mycolor!75, opacity=1] ({0.01*\b*cos(\a)}, {0.01*\b*sin(\a)}) circle (1.5pt);
        };                          
%Verkabelung
\node[draw=none,fill=none] at (-4.35, 0.1){$U_\mathrm{Heiz}$};
\node[draw=none,fill=none] at (-3.9, 0.4){$+$};
\node[draw=none,fill=none] at (-3.9, -0.20){$-$};
\draw[thick] (-1.8,0.0) -- ++ (-2.1,0);
\draw[thick] (-1.04,0.0)--(-1.04,0.2) -- ++ (-2.86,0) ;
\draw [fill=white] (-3.9, 0.0) circle (2pt);
\draw [fill=white] (-3.9, 0.2) circle (2pt);
\draw[fill=black] (-3.4,0.0) circle (2pt);
\draw[thick] (-3.4,0.0) -- ++ (0,-0.25);
\draw[thick] (-3.2,-0.25) -- ++ (1.4,0) -- ++ (0,-0.5);
\draw [fill=white] (-3.2, -0.25) circle (2pt);
\draw [fill=white] (-3.4, -0.25) circle (2pt);
\node[draw=none,fill=none] at (-3.3, -0.65){$U_\mathrm{B}$};
\node[draw=none,fill=none] at (-3.2, -0.5){$+$};
%Elektronenstrahl
 \draw[postaction={decorate},decoration={markings,mark=at position 0.33 with {\arrow{Triangle[length=0pt 3*4,width=0pt 4]}}},
            decoration={markings,mark=at position 0.67 with {\arrow{Triangle[length=0pt 3*4,width=0pt 4]}}}, mycolor, thick] (-1.42,-0.85) arc (210.9:522:1.69);
%Kathode
\draw[decoration={aspect=0.9, segment length=3.4, amplitude=1.5,coil},decorate, mycolor3, thick] (-1.8,0.0) -- (-1.04, 0.0);
% Anode
\draw[fill=mycolor2!50] (-1.8,-0.7) arc (180:540:0.38 and 0.1);
\draw[fill=mycolor2!50] (-1.8,-0.8) arc (180:360:0.38 and 0.1) -- ++ (0.0,0.1) arc (360:900:0.38 and 0.1) -- cycle;
\draw[fill=halfgray!20, opacity=0.7] (-1.55,-0.70) arc (180:-180:0.12 and 0.04) arc (180:210:0.12 and 0.04) arc (150:30:0.12 and 0.04) ;
% Schutzblech 
\draw[fill=halfgray!50]  (-0.95, 0.45) -- ++ (-0.05, -0.05) -- ++ (-0.85, 0)-- ++ (0, 0.05) -- ++ (0.1, 0.1) -- ++ (0.9,0)  -- ++ (-0.10, -0.10) -- ++ (-0.9, 0);
\draw[fill=halfgray!50] (-0.95, 0.45) -- (-1.0, 0.40) -- ++ (0, -1.40) -- ++ (0.05, 0) -- ++ (0, 1.45) -- ++ (0.1,0.1) -- ++ (0, -1.45)-- ++ (-0.1, -0.1);
\draw[fill=halfgray!50] (-1.85, 0.45) -- (-0.95, 0.45) -- (-0.95, -1) -- (-1, -1) -- (-1, 0.4) -- (-1.85, 0.4) -- cycle;
% Drehwurm aka Vakuumröhre
\draw[mycolor!50] (-1.93,0.52) arc (165:-165:2) arc (15:90:0.29) -- ++ (-0.8,0) arc (270:90:0.1 and 0.31) arc (90:-90:0.1 and 0.31) arc (270:90:0.1 and 0.31) -- ++ (0.80,0) arc (270:345:0.29) ;
\shade[bottom color=mycolor!10, top color=mycolor!50,opacity=0.20] (-1.93,0.52) arc (165:-165:2) arc (15:90:0.29) -- ++ (-0.8,0) arc (270:90:0.1 and 0.31) arc (90:-90:0.1 and 0.31) arc (270:90:0.1 and 0.31) -- ++ (0.80,0) arc (270:345:0.29) ;
% Elektronenstrahl
\draw[->,>={Triangle[length=0pt 3*3,width=0pt 3]}, thick, mycolor, shorten >=0.4pt] (-1.42, -.1)  -- (-1.42, -.75);
%Helmholtz-Spule
\path[top color=halfgray!75, bottom color=halfgray!50, draw=black] (0, 0) circle[radius=2.6] circle[radius=2.3];
%Kraft
  \draw[->,>={Triangle[length=0pt 3*4,width=0pt 4]},mycolor4, thick]  (0.29,1.70)   -- ++ (-0.98, 0.17); 
   \draw[->,>={Triangle[length=0pt 3*4,width=0pt 4]},mycolor4!50!mycolor2, thick]  (0.29,1.70)   -- ++ (-0.17, -0.98); 
  \shade[ball color=mycolor!25, opacity=1] (0.29,1.70) circle (3pt);
    \node at (0.29,1.70) {$-$};   
  \draw[->,>={Triangle[length=0pt 3*4,width=0pt 4]},mycolor4, thick]  (0.43,-1.63)   -- ++ (0.97, 0.26); 
      \draw[<->,>={Triangle[length=0pt 3*4,width=0pt 4]}, thick]  (0.43,-1.63)   -- ++ (-0.44, 1.63); 
      \node at (0.31, -.81){$r$};
  \shade[ball color=mycolor!25, opacity=1] (0.43,-1.63) circle (3pt);
    \node at (0.43,-1.63) {$-$};  
        \node at (-0.2,1.92) {$\boldsymbol{\dot{r}}$};  
            \node at (0.88,-1.62) {$\boldsymbol{\dot{r}}$};  
          \node at (0.12,0.64) {$\boldsymbol{F}_\mathrm{L}=\boldsymbol{F}_\mathrm{z}$};  
                
\end{scope}
\end{tikzpicture}
  \caption[Das Fadenstrahlrohr-Experiment]{Das Fadenstrahlrohr-Experiment zur Bestimmung der Elektronenmasse kann als Anschauung für das Verhalten bewegter Ladungen in Magnetfeldern genutzt werden. Eine Elektronenkanone, bestehend aus Glühkathode und Lochanode beschleunigt Elektronen, welche bei eingeschaltetem Helmholtz-Spulenpaar, dessen B-Feld aus der Papierebene herauszeigt, durch die Bedingung $\boldsymbol{F}_\mathrm{L}=\boldsymbol{F}_\mathrm{z}$ auf eine Kreisbahn gezwungen werden. Ein Edelgas bei niedrigem Druck (so dass die mittlere freie Weglänge ausreichend groß ist) wird durch Stoßionisation auf der Kreisbahn zum Leuchten gebracht (eigene Darstellung).}
  \label{fig:fade}
  \vspace{-0pt}
\end{figure}