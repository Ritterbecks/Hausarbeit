  \thisfloatsetup{%
  capbesidewidth=\marginparwidth}

\begin{figure}[htbp]
\centering
%\sansmath
\begin{tikzpicture}
\begin{axis}[	
	clip=false, % Verhindet weiße Punkte bei vielfach geplotteten x-Werten
	ymajorgrids,
    xmajorgrids,
          axis x line*=middle,
          axis y line*=center,
    grid style={white},
	axis on top,
    width=12cm,
    height=7.797cm,
    xmin=0,
    xmax=1200,
    ymin=-150,
    ymax=200,
    %/tikz/ybar interval,
    tick align=center,
    xtick align=outside,
    xlabel={$x$},
    ylabel={\rotatebox[origin=b]{270}{$y$}},   
   x label style={at={(axis cs:1000,-45)}},
    y label style={at={(axis cs:-150.5,150)}},
    axis line style={Honeydew4!70!black},
    ticklabel style={Honeydew4!70!black, inner sep=1pt,
                font=\footnotesize},
    yticklabels={${-150}$, ${-100}$, ${-50}$,  ${0}$,   ${50}$ , ${100}$, ${\si{\milli\newton}}$, ${200}$ },            
    xtick={200,400,600,800,1000, 1200},
    xticklabels={${200}$, ${400}$, ${600}$,  ${800}$, ${\si{\milli\metre}}$, ${1200}$},
     ytick={-150, -100, -50, 0, 50, 100,150,200 }, 
    scaled ticks=false,
    width=\textwidth,        
       legend style ={ at={(axis cs:150,-30)}, 
         anchor=north west, draw=none, 
             fill=none,align=left, text=Honeydew4!70!black,
              font=\footnotesize},                      
    cycle list name=mycolor4 white,                    
    label style={font=\small,Honeydew4!70!black,},
    enlarge x limits=true,
    %tick style={draw=none},
    x tick label as interval=false,
]
\draw[thick, Honeydew4!70!black, ->,>={latex[Honeydew4!70!black, round, length=0.4cm, width=4pt]}] (axis cs:-150.50, 110) -- (axis cs:-150.50, 190);
\draw[thick,Honeydew4!70!black, ->,>={latex[Honeydew4!70!black,round, length=0.4cm, width=4pt]}] (axis cs:900.00, -45) -- (axis cs:1100.00, -45);
\draw[thick, mycolor!50, fill=mycolor] (axis cs:675,18) circle (3pt);
\addplot
table[x = x, y =y]{images/data/10.dat};
\addlegendentry{$\SI{22.0}{\milli\metre}$};
\addplot
table[x = x, y =y]{images/data/11.dat};
\addlegendentry{$\SI{24.0}{\milli\metre}$};
\addplot
table[x = x, y =y]{images/data/12.dat};
\addlegendentry{$\SI{35.0}{\milli\metre}$};
\addplot
table[x = x, y =y]{images/data/13.dat};
\addlegendentry{$\SI{47.0}{\milli\metre}$};
\addplot
table[x = x, y =y]{images/data/14.dat};
\addlegendentry{$\SI{60.0}{\milli\metre}$};
\end{axis}
\clipright
\end{tikzpicture}
  \caption[Plot der theoretisch berechneten Kugelbahnen]{Plots der theoretisch berechneten Kugelbahnen für Starthöhen $h$, die mit denjenigen der Videoanalysen nahezu übereinstimmen. Der blaue Kreis deutet die approximative Fokussierungszone an (bei $(x,y)\tran = [(675, 19)\tran \pm (10 , 2)\tran] \si{\milli\metre}$).}
  \label{fig:plot1}
  \vspace{-0pt}
\end{figure}