  \thisfloatsetup{%
  capbesidewidth=\marginparwidth}

\begin{figure}[htbp]
\centering
%\sansmath
\subfloat[Stufe 2]{
\begin{tikzpicture}
\begin{axis}[	
	clip=false, % Verhindet weiße Punkte bei vielfach geplotteten x-Werten
	ymajorgrids,
    xmajorgrids,
          axis x line*=middle,
          axis y line*=center,
    grid style={white},
	axis on top,
    width=12cm,
    height=4.797cm,
    xmin=-7,
    xmax=7,
    ymin=0,
    ymax=350,
    %/tikz/ybar interval,
    tick align=center,
    xtick align=outside,
    xlabel={Abstand $d$},
    ylabel={\rotatebox[origin=b]{270}{Windlast $F_\mathrm{LW}$}},   
   x label style={at={(axis cs:5,-65)}},
    y label style={at={(axis cs:-3.5,300)}},
    axis line style={Honeydew4!70!black},
    ticklabel style={Honeydew4!70!black, inner sep=1pt,
                font=\footnotesize},
    yticklabels={${50}$, ${100}$, ${150}$,  ${200}$,   ${250}$ , ${\si{\milli\newton}}$, ${350}$ },            
    ytick={50,100,150,200, 250, 300, 350},
    xticklabels={-7,-5,-3,-1,1,3, $\si{\centi\metre}$, 7},
     xtick={-7,-5,-3,-1,1,3,5, 7}, 
    scaled ticks=false,
    width=\textwidth,                                   
    label style={font=\small,Honeydew4!70!black,},
    enlarge x limits=true,
    %tick style={draw=none},
    x tick label as interval=false,
]
\draw[thick, Honeydew4!70!black, ->,>={latex[Honeydew4!70!black, round, length=0.4cm, width=4pt]}] (axis cs:-3.50, 260) -- (axis cs:-3.50, 340);
\draw[thick,Honeydew4!70!black, ->,>={latex[Honeydew4!70!black,round, length=0.4cm, width=4pt]}] (axis cs:3.00, -65) -- (axis cs:7.00, -65);
\addplot[fill=mycolor4, draw=mycolor4, only marks, mark=*, error bars/.cd,
   x dir=both,
   x explicit, error bar style={mycolor4}]
table[x = x, y =y, x error=dx]{images/data/kraftmessung1S2.dat};
\addplot[mycolor, fill=mycolor!25, smooth, domain=-8:8,thick, opacity=0.4] {302.8*exp(-((x)/2.143)^2)};
\end{axis}
\clipright
\end{tikzpicture}}
\\
\subfloat[Stufe 1]{
\begin{tikzpicture}
\begin{axis}[	
	clip=false, % Verhindet weiße Punkte bei vielfach geplotteten x-Werten
	ymajorgrids,
    xmajorgrids,
          axis x line*=middle,
          axis y line*=center,
    grid style={white},
	axis on top,
    width=12cm,
    height=4.797cm,
    xmin=-7,
    xmax=7,
    ymin=0,
    ymax=350,
    %/tikz/ybar interval,
    tick align=center,
    xtick align=outside,
    xlabel={Abstand $d$},
    ylabel={\rotatebox[origin=b]{270}{Windlast $F_\mathrm{LW}$}},   
   x label style={at={(axis cs:5,-65)}},
    y label style={at={(axis cs:-3.5,300)}},
    axis line style={Honeydew4!70!black},
    ticklabel style={Honeydew4!70!black, inner sep=1pt,
                font=\footnotesize},
    yticklabels={${50}$, ${100}$, ${150}$,  ${200}$,   ${250}$ , ${\si{\milli\newton}}$, ${350}$ },            
    ytick={50,100,150,200, 250, 300, 350},
    xticklabels={-7,-5,-3,-1,1,3, $\si{\centi\metre}$, 7},
     xtick={-7,-5,-3,-1,1,3,5, 7}, 
    scaled ticks=false,
    width=\textwidth,                                   
    label style={font=\small,Honeydew4!70!black,},
    enlarge x limits=true,
    %tick style={draw=none},
    x tick label as interval=false,
]
\draw[thick, Honeydew4!70!black, ->,>={latex[Honeydew4!70!black, round, length=0.4cm, width=4pt]}] (axis cs:-3.50, 260) -- (axis cs:-3.50, 340);
\draw[thick,Honeydew4!70!black, ->,>={latex[Honeydew4!70!black,round, length=0.4cm, width=4pt]}] (axis cs:3.00, -65) -- (axis cs:7.00, -65);
\addplot[fill=mycolor, draw=mycolor, only marks, mark=*, error bars/.cd,
   x dir=both,
   x explicit, error bar style={mycolor}]
table[x = x, y =y, x error=dx]{images/data/kraftmessung1S1.dat};
\addplot[mycolor4, fill=mycolor4!25, smooth, domain=-8:8,thick, opacity=0.4] {188.1*exp(-((x)/2.133)^2)};
\end{axis}
\clipright
\end{tikzpicture}}
  \caption[Beispielhafter Gaußglocken-Fit]{Beispiel eines in \textit{Matlab} erzeugten Gaußglocken-Fits für die Messung der Kugel mit $r=\SI{17.5}{\milli\metre}$ bei einem Abstand von $s=(1.0\pm 0.1)\,\si{\centi\metre}$. Als Anpassungsfunktion wird für alle Messreihen $f(d)=A\cdot \exp(-\left(\sfrac{d-\mu}{2\sigma^2}\right)^2)$ verwendet, da sie im Mittel die geringsten quadratischen Abweichungen bei den vorhandenen Daten liefert. Alle graphischen Auftragungen der Messwerte wurden um das jeweilige $\mu$ verschoben. Die Fehlerbalken sind kaum zu erkennen, da als Unsicherheit auf den Abstand des Gebläses von der Mittelachse ein Winkel von $\varphi=\SI{2}{\degree}$ angenommen wird und damit die Messungenauigkeit linear mit dem Abstand des Föhns skaliert. }
  \label{fig:kraftmessung1}
  \vspace{-0pt}
\end{figure}