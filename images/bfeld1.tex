%% Autor: Björn Ritterbecks 
%% Letzte Aenderung: 15.06.2016 
\thisfloatsetup{%
  capbesidewidth=\marginparwidth}
\begin{figure}[htbp]
\centering
\usetikzlibrary{decorations.pathmorphing}
\pgfplotsset{width=7cm,compat=1.13}
\small
%\sansmath
\subfloat[\label{fig:bfelda}]{
\begin{tikzpicture}
\begin{scope}[scale=1.12]
% Stabmagnet überdeckt
\draw [fill=halfgray!25] (1.6, 0.4) -- ++ (0, -0.6) -- ++ (-0.1, -0.1) -- ++ (0, 0.6);
% Feldlinien
% Norden
\draw[postaction={decorate},decoration={markings,mark=at position 0.70 with {\arrow{Triangle[length=0pt 3*4,width=0pt 4]}}}, mycolor2, thick]  (-1.45, 0.05) arc (270:259:8);
\draw[postaction={decorate},decoration={markings,mark=at position 0.60 with {\arrow{Triangle[length=0pt 3*4,width=0pt 4]}}}, mycolor2, thick]  (-1.45, 0.11) arc (265:242:4);
\draw[postaction={decorate},decoration={markings,mark=at position 0.50 with {\arrowreversed{Triangle[length=0pt 3*4,width=0pt 4]}}}, mycolor2, thick]  (1.55, 0.05) arc (270:281:8);
\draw[postaction={decorate},decoration={markings,mark=at position 0.40 with {\arrowreversed{Triangle[length=0pt 3*4,width=0pt 4]}}}, mycolor2, thick]  (1.55, 0.11) arc (275:298:4);
 \draw[postaction={decorate},decoration={markings,mark=at position 0.33 with {\arrow{Triangle[length=0pt 3*4,width=0pt 4]}}},
            decoration={markings,mark=at position 0.67 with {\arrow{Triangle[length=0pt 3*4,width=0pt 4]}}}, mycolor2, thick] (-1.45,0.35) arc (205:-25:1.65 and 0.5) ;
  \draw[postaction={decorate},decoration={markings,mark=at position 0.33 with {\arrow{Triangle[length=0pt 3*4,width=0pt 4]}}},
             decoration={markings,mark=at position 0.67 with {\arrow{Triangle[length=0pt 3*4,width=0pt 4]}}}, mycolor2, thick] (-1.45,0.29) arc (220:-40:1.96 and 0.7) ;   
  \draw[postaction={decorate},decoration={markings,mark=at position 0.33 with {\arrow{Triangle[length=0pt 3*4,width=0pt 4]}}},
             decoration={markings,mark=at position 0.67 with {\arrow{Triangle[length=0pt 3*4,width=0pt 4]}}}, mycolor2, thick] (-1.45,0.23) arc (233:-53:2.5 and 0.9);      
  \draw[postaction={decorate},decoration={markings,mark=at position 0.33 with {\arrow{Triangle[length=0pt 3*4,width=0pt 4]}}},
             decoration={markings,mark=at position 0.67 with {\arrow{Triangle[length=0pt 3*4,width=0pt 4]}}}, mycolor2, thick] (-1.45,0.17) arc (240:-60:3.0 and 1.1) ;    
% Süden
\draw[postaction={decorate},decoration={markings,mark=at position 0.70 with {\arrow{Triangle[length=0pt 3*4,width=0pt 4]}}}, mycolor2, thick]  (-1.45, -0.00) arc (90:101:8);
\draw[postaction={decorate},decoration={markings,mark=at position 0.60 with {\arrow{Triangle[length=0pt 3*4,width=0pt 4]}}}, mycolor2, thick]  (-1.45, -0.06) arc (95:118:4);
\draw[postaction={decorate},decoration={markings,mark=at position 0.50 with {\arrowreversed{Triangle[length=0pt 3*4,width=0pt 4]}}}, mycolor2, thick]  (1.55, -0.00) arc (90:79:8);
\draw[postaction={decorate},decoration={markings,mark=at position 0.40 with {\arrowreversed{Triangle[length=0pt 3*4,width=0pt 4]}}}, mycolor2, thick]  (1.55, -0.06) arc (85:62:4);
 \draw[postaction={decorate},decoration={markings,mark=at position 0.33 with {\arrow{Triangle[length=0pt 3*4,width=0pt 4]}}},
            decoration={markings,mark=at position 0.67 with {\arrow{Triangle[length=0pt 3*4,width=0pt 4]}}}, mycolor2, thick] (-1.45,-0.29) arc (155:385:1.65 and 0.5) ;
  \draw[postaction={decorate},decoration={markings,mark=at position 0.33 with {\arrow{Triangle[length=0pt 3*4,width=0pt 4]}}},
             decoration={markings,mark=at position 0.67 with {\arrow{Triangle[length=0pt 3*4,width=0pt 4]}}}, mycolor2, thick] (-1.45,-0.23) arc (140:400:1.96 and 0.7) ;   
  \draw[postaction={decorate},decoration={markings,mark=at position 0.33 with {\arrow{Triangle[length=0pt 3*4,width=0pt 4]}}},
             decoration={markings,mark=at position 0.67 with {\arrow{Triangle[length=0pt 3*4,width=0pt 4]}}}, mycolor2, thick] (-1.45,-0.17) arc (127:413:2.5 and 0.9);      
  \draw[postaction={decorate},decoration={markings,mark=at position 0.33 with {\arrow{Triangle[length=0pt 3*4,width=0pt 4]}}},
             decoration={markings,mark=at position 0.67 with {\arrow{Triangle[length=0pt 3*4,width=0pt 4]}}}, mycolor2, thick] (-1.45,-0.11) arc (120:420:3.0 and 1.1) ;               
% Stabmagnet 
\shade[left color=halfgray!75, right color=halfgray!25,opacity=1]
(-1.5,-0.3) rectangle (1.5,0.3);
\shade[left color=halfgray!75, right color=halfgray!25,opacity=1]
(-1.5,0.3) -- (-1.4,0.4) -- ++ (3,0) -- ++ (-0.1,-0.1) -- ++ (-3, 0);
\draw (-1.5, 0.3) -- (-1.4, 0.4) -- (1.6, 0.4) -- ++ (-0.1, -0.1) -- ++ (0, -0.6) -- ++ (0.1, 0.1) -- ++ (0, 0.6);
\draw (-1.5, -0.3) rectangle (1.5, 0.3);
\node[draw=none,fill=none] at (-1.3, 0.0){\footnotesize N};
\node[draw=none,fill=none] at (1.3, 0.0){\footnotesize S};
\end{scope}
\end{tikzpicture}}
\qquad
\subfloat[\label{fig:bfeldb}]{
\begin{tikzpicture}[
	scale=1,
	ka roehre/.style={fill=white,draw=black!80}
]
\begin{scope}[scale=1.21]
% Nordpol
\draw[fill=halfgray!50] (-4.9,1.5) -- ++ (0.0,1) arc (180:360:0.55 and 0.15) arc (0:180:0.55 and 0.15) arc (180:360:0.55 and 0.15) -- ++ (0.0,-1) decorate[decoration={random steps,segment length=1.5pt,amplitude=0.5pt}] {arc (360:180:0.55 and 0.15) };
\node[draw=none,fill=none] at (-4.35, 2.0){\footnotesize S};
% Feldlinien
 \draw[postaction={decorate},decoration={markings,mark=at position 0.41 with {\arrow{Triangle[length=0pt 3*4,width=0pt 4]}}},
            decoration={markings,mark=at position 0.8 with {\arrow{Triangle[length=0pt 3*4,width=0pt 4]}}}, mycolor2, thick] (-4.35,4.2) -- ++ (0,-1.7) ;
 \draw[postaction={decorate},decoration={markings,mark=at position 0.41 with {\arrow{Triangle[length=0pt 3*4,width=0pt 4]}}},
            decoration={markings,mark=at position 0.8 with {\arrow{Triangle[length=0pt 3*4,width=0pt 4]}}}, mycolor2, thick] (-4.75,4.2) arc (168:192:4.1) ;
 \draw[postaction={decorate},decoration={markings,mark=at position 0.41 with {\arrow{Triangle[length=0pt 3*4,width=0pt 4]}}},
            decoration={markings,mark=at position 0.8 with {\arrow{Triangle[length=0pt 3*4,width=0pt 4]}}}, mycolor2, thick] (-3.95,4.2) arc (12:-12:4.1) ;                        
   \draw[postaction={decorate},decoration={markings,mark=at position 0.41 with {\arrow{Triangle[length=0pt 3*4,width=0pt 4]}}},
              decoration={markings,mark=at position 0.8 with {\arrow{Triangle[length=0pt 3*4,width=0pt 4]}}}, mycolor2, thick] (-4.55,4.2) arc (174:186:8.15) ;
   \draw[postaction={decorate},decoration={markings,mark=at position 0.41 with {\arrow{Triangle[length=0pt 3*4,width=0pt 4]}}},
              decoration={markings,mark=at position 0.8 with {\arrow{Triangle[length=0pt 3*4,width=0pt 4]}}}, mycolor2, thick] (-4.15,4.2) arc (6:-6:8.15) ;   
 \draw[postaction={decorate},decoration={markings,mark=at position 0.41 with {\arrow{Triangle[length=0pt 3*4,width=0pt 4]}}},
            decoration={markings,mark=at position 0.8 with {\arrow{Triangle[length=0pt 3*4,width=0pt 4]}}}, mycolor2, thick] (-4.9,4.2) arc (150:210:1.7) ;
 \draw[postaction={decorate},decoration={markings,mark=at position 0.41 with {\arrow{Triangle[length=0pt 3*4,width=0pt 4]}}},
            decoration={markings,mark=at position 0.8 with {\arrow{Triangle[length=0pt 3*4,width=0pt 4]}}}, mycolor2, thick] (-3.8,4.2) arc (30:-30:1.7) ;                       
\draw[fill=halfgray!75] (-4.9,5.22) -- ++ (0.0,-1) arc (180:360:0.55 and 0.15) -- ++ (0.0,1) decorate[decoration={random steps,segment length=1.5pt,amplitude=0.5pt}] {arc (0:180:0.55 and 0.15) arc (180:360:0.55 and 0.15)};
\node[draw=none,fill=none] at (-4.35, 4.72){\footnotesize N};
\end{scope}
\end{tikzpicture}}
  \caption[Magnetfeld eines Stabmagneten und eines offenen Ringmagneten]{Die magnetischen Feldlinien laufen per definitionem vom Nord- zum Südpol und sind im Allgemeinen geschlossen, woraus $\div \boldsymbol{B}=0$ folgt. {\color{mycolor}\textbf{(a)}:} Die Feldlinien des Stabmagneten laufen im Inneren des Volumens weiter. {\color{mycolor}\textbf{(b)}:} Zwischen den großen Polschuhen mit Durchmesser $D>>d$ ($d$ als Durchmesser des Hufeisenmagneten) herrscht ein näherungsweise homogenes Magnetfeld $\boldsymbol{B}$ (eigene Darstellung).}
  \label{fig:bfeld1}
  \vspace{-0pt}
\end{figure}