  \thisfloatsetup{%
  capbesidewidth=\marginparwidth}

\begin{figure}[htb!p]
\centering
\pgfplotsset{compat=1.3}
%\sansmath
\subfloat[]{
\begin{tikzpicture}
\begin{axis}[	
	clip=false,
%clip=false, % Verhindet weiße Punkte bei vielfach geplotteten x-Werten
	ymajorgrids,
    xmajorgrids,
              axis x line*=middle,
              axis y line*=none,
    grid style={white,thick},
	axis on top,
    width=11cm,
    height=6.797cm,
    xmin=-12,
    xmax=75,
    ymin=0,
    ymax=0.8,
    /tikz/ybar interval,
    tick align=outside,
    xlabel={$\varphi$},
    x label style={at={(axis cs:60.75,-0.1)}},
    y label style={at={(axis cs:-13.75,0.5)}},
    ylabel={\rotatebox[origin=b]{270}{$\rho$}},    
    axis line style={draw opacity=0},
    ticklabel style={Honeydew4!70!black, inner sep=1pt,
                font=\footnotesize},
    yticklabels={${0,0}$, ${0,2}$, ${0,4}$, $\si{\percent}$, ${0,8}$ },            
    ytick={0.0,0.2,0.4,0.6, 0.8},
    xticklabels={$ $,$\SI{-6.75}{\degree}$, $$, $$, $\SI{6.75}{\degree}$, $$, $$, $\SI{20.25}{\degree}$, $$,$$,
                     $\SI{33.75}{\degree}$, $$,$$, $\SI{47.25}{\degree}$, $$,$$, $\SI{60.75}{\degree}$,
                     $$,$$, $\SI{74.25}{\degree}$},
           scaled ticks=false,
    width=\textwidth,                        
    xtick=data,                          
    label style={font=\small, Honeydew4!70!black},
    enlarge x limits=true,
    tick style={draw=none},
    x tick label as interval=false,
    nodes near coords={\pgfmathfloatifflags{\pgfplotspointmeta}{0}{}{\pgfmathprintnumber{\pgfplotspointmeta}}},
    every node near coord/.append style={
    fill=white,
    /pgf/number format/precision=2,
    /pgf/number format/fixed zerofill,
    anchor=mid west,    
    shift={(3pt,4pt)},
    inner sep=0,
    above,
    font=\footnotesize,
    rotate=45},
           legend style ={ at={(axis cs:6.75,0.7)}, 
                anchor=north west, draw=none, 
                fill=none,align=left, text=Honeydew4!70!black, font=\footnotesize},
]
\draw[thick,Honeydew4!70!black, ->,>={Kite[round, length=0.4cm, width=4pt]}] (axis cs:-13.75,0.4) -- (axis cs:-13.75,0.6);
\draw[thick, Honeydew4!70!black,->,>={Kite[round, length=0.4cm, width=4pt]}] (axis cs:54.00,-0.1) -- (axis cs:72.00, -0.1);
\addplot[mycolor2!70!white, fill=mycolor2, draw=none, mark=none]
table[x =Lower, y =Count,]{images/data/Mais0g.dat};
 \addplot[mycolor!70!white, fill=mycolor, draw=none, mark=none]
  table[x =Lower, y =Count,]{images/data/Mais1g.dat};
 \addplot[mycolor4!70!white, fill=mycolor4, draw=none, mark=none]
   table[x =Lower, y =Count,]{images/data/Mais3g.dat};
\end{axis}
\clipright
\end{tikzpicture}}
\\
\subfloat[]{
\begin{tikzpicture}
\begin{axis}[	
	clip=false,
%clip=false, % Verhindet weiße Punkte bei vielfach geplotteten x-Werten
	ymajorgrids,
    xmajorgrids,
              axis x line*=middle,
              axis y line*=none,
    grid style={white,thick},
	axis on top,
    width=11cm,
    height=6.797cm,
    xmin=-12,
    xmax=75,
    ymin=0,
    /tikz/ybar interval,
    tick align=outside,
    xlabel={$\varphi$},
    x label style={at={(axis cs:60.75,-0.1)}},
    y label style={at={(axis cs:-13.75,0.5)}},
    ylabel={\rotatebox[origin=b]{270}{$\rho$}},    
    axis line style={draw opacity=0},
    ticklabel style={Honeydew4!70!black, inner sep=1pt,
                font=\footnotesize},
    yticklabels={${0,0}$, ${0,2}$, ${0,4}$, $\si{\percent}$, ${0,8}$ },            
    ytick={0.0,0.2,0.4,0.6, 0.8},
    xticklabels={$ $,$\SI{-6.75}{\degree}$, $$, $$, $\SI{6.75}{\degree}$, $$, $$, $\SI{20.25}{\degree}$, $$,$$,
                     $\SI{33.75}{\degree}$, $$,$$, $\SI{47.25}{\degree}$, $$,$$, $\SI{60.75}{\degree}$,
                     $$,$$, $\SI{74.25}{\degree}$},
           scaled ticks=false,
    width=\textwidth,                        
    xtick=data,                          
    label style={font=\small, Honeydew4!70!black},
    enlarge x limits=true,
    tick style={draw=none},
    x tick label as interval=false,
    nodes near coords={\pgfmathfloatifflags{\pgfplotspointmeta}{0}{}{\pgfmathprintnumber{\pgfplotspointmeta}}},
    every node near coord/.append style={
    fill=white,
    /pgf/number format/precision=2,
    /pgf/number format/fixed zerofill,
    anchor=mid west,    
    shift={(3pt,4pt)},
    inner sep=0,
    above,
    font=\footnotesize,
    rotate=45},
           legend style ={ at={(axis cs:6.75,0.7)}, 
                anchor=north west, draw=none, 
                fill=none,align=left, text=Honeydew4!70!black, font=\footnotesize},
]
\draw[thick,Honeydew4!70!black, ->,>={Kite[round, length=0.4cm, width=4pt]}] (axis cs:-13.75,0.4) -- (axis cs:-13.75,0.6);
\draw[thick, Honeydew4!70!black,->,>={Kite[round, length=0.4cm, width=4pt]}] (axis cs:54.00,-0.1) -- (axis cs:72.00, -0.1);
\addplot[mycolor2!70!white, fill=mycolor2, draw=none, mark=none]
table[x =Lower, y =Count,]{images/data/nmDet.dat};
 \addplot[mycolor!70!white, fill=mycolor, draw=none, mark=none]
  table[x =Lower, y =Count,]{images/data/3cmleerDet.dat};
 \addplot[mycolor4!70!white, fill=mycolor4, draw=none, mark=none]
   table[x =Lower, y =Count,]{images/data/3cmrotDet.dat};
\end{axis}
\clipright
\end{tikzpicture}}
  \caption[Aufspaltung mit Föhn als Analysator]{
  {\color{mycolor}\textbf{(a)}:} Referenzmessung von \textsc{Mais} für jeweils $n=50$ Kugeln mit --- von links nach rechts --- $\SI{0}{\gram}$, $\SI{1}{\gram}$ und $\SI{3}{\gram}$ Eisenwollefüllung. Die Messreihe mit einer Füllung von $\SI{2}{\gram}$ wird aufgrund zu vieler Überlappungen nicht dargestellt.
   {\color{mycolor}\textbf{(b)}:}Drei Messreihen zur Tauglichkeit des Föhns als Massenanalysator mit jeweils $n=50$ Datensätzen. Von links nach rechts: Ohne Föhn, befüllte Kugeln, leere Kugeln. Der Hohe Ausschlag bei $x=\SI{0}{\degree}$ lässt sich durch die Videoanalyse erklären: Selbst bei einer $y$-Koordinate von $\SI{2.24}{\centi\metre}$ wird eine Kugel noch als im Intervall $(-2,25; 2,25]$ detektiert gewertet. An Stelle einer tabellarischen Zusammenfassung sind alle relativen Häufigkeiten an den Histogrammen vermerkt.}
  \label{fig:maisbex}
  \vspace{-0pt}
\end{figure}