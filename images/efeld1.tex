%% Autor: Björn Ritterbecks 
%% Letzte Aenderung: 15.06.2016 
\thisfloatsetup{%
  capbesidewidth=\marginparwidth}
\begin{figure}[htbp]
\centering
\usetikzlibrary{decorations.pathmorphing}
\pgfplotsset{width=7cm,compat=1.13}
\small
%\sansmath
\begin{tikzpicture}[
	scale=1,
	ka roehre/.style={fill=white,draw=black!80}
]
\begin{scope}[scale=2.17]

%Verkabelung
\node[draw=none,fill=none] at (-5.30, 3.55){$-$};
\node[draw=none,fill=none] at (-5.30, 2.65){$+$};
\node[draw=none,fill=none] at (-5.48, 3.7){\footnotesize $y$};
\node[draw=none,fill=none] at (-2.71, 3.00){\footnotesize $x$};
\node[draw=none,fill=none] at (-6.2, 2.8){\footnotesize $z$};
%Achsen
\draw[->,>={Triangle[length=0pt 3*4,width=0pt 4]}, thick] (-5.40, 2.6) -- (-5.40, 3.8);
\draw[->,>={Triangle[length=0pt 3*4,width=0pt 4]}, thick] (-7.51, 3.1) -- (-2.61, 3.1);
%Länge
\draw[<->,>={Triangle[length=0pt 3*4,width=0pt 4]}, mycolor2] (-5.9, 2.5) -- (-4.8, 2.5);
\draw[mycolor2] (-5.9, 2.4) -- (-5.9, 2.6);
\draw[mycolor2] (-4.8, 2.4) -- (-4.8, 2.6);
\node[rectangle,fill=white] at (-5.40, 2.50){\footnotesize $l$};
%E-Feld
\draw[fill=mycolor!75] (-6.0, 2.8) -- (-5.8, 2.9) -- (-4.7, 2.9) -- (-4.9, 2.8);   \draw[fill=mycolor!75] (-6.0, 2.8) -- (-6.0, 2.75) -- (-4.9, 2.75) -- (-4.7, 2.85) -- (-4.7, 2.9) -- (-4.9, 2.8) -- (-6.0, 2.8);     
\draw (-4.9, 2.75) -- (-4.9, 2.8);
\foreach \x in {0,...,7}
        {
\draw[->,>={Triangle[length=0pt 3*3,width=0pt 3]}, thick, mycolor, shorten >=0.4pt] ({-5.87+(\x*0.15)},2.85) -- ({-5.87+(\x*0.15)},3.40) ;
 };
\draw[fill=mycolor!25] (-6.0, 3.4) -- (-5.8, 3.5) -- (-4.7, 3.5) -- (-4.9, 3.4);
\draw[fill=mycolor!25] (-6.0, 3.4) -- (-6.0, 3.35) -- (-4.9, 3.35) -- (-4.7, 3.45) -- (-4.7, 3.5) -- (-4.9, 3.4) -- (-6.0, 3.4); 
\draw (-4.9, 3.35) -- (-4.9, 3.4);
\begin{scope}[thick, every node/.style={sloped,allow upside down}]
% Elektronenbahn
\draw[->,>={Triangle[length=0pt 3*3,width=0pt 3]},mycolor, thick] (-7.51, 3.1)  -- node {\midarrow} (-6.0, 3.1) arc (90:70:4.2)  -- node {\midarrow} ++ (0.6,-0.22) -- ++ (1.2,-0.44); 
%Geschwindigkeit
\draw[dotted, mycolor4]  (-4.0,2.64)   -- ++ (0.7,0) -- ++ (0.0,-0.26) -- ++ (-0.7,0) ;
    \draw[->,>={Triangle[length=0pt 3*4,width=0pt 4]},mycolor4, thick]  (-6.63,3.1)  -- (-5.93, 3.1) ;
       \node at (-6.28,3.2) {\footnotesize $\dot{x}$};    
        \draw[->,>={Triangle[length=0pt 3*4,width=0pt 4]},mycolor4, thick]  (-4.0,2.64)  --  (-3.3, 2.64) ;  
   \node at (-3.65,2.74) {\footnotesize $\dot{x}$};       
        \draw[->,>={Triangle[length=0pt 3*4,width=0pt 4]},mycolor4, thick]  (-4.0,2.64)  -- ++ (0.7, -0.26) ;
        \node at (-3.35,2.3) {\footnotesize $(\dot{x}, \dot{y})\tran$};                   
        \draw[->,>={Triangle[length=0pt 3*4,width=0pt 4]},mycolor4, thick]  (-4.0,2.64)  -- ++ (0.0, -0.26) ; 
        \node at (-4.1,2.51) {\footnotesize $\dot{y}$}; 
%Winkel
\draw[mycolor2, thick]  (-4.56,2.85)   -- (-4.0,2.85) ;
\draw[mycolor2, thick]  (-4.06,2.85) arc (360:339.86:0.5) ;
\node at (-4.2,2.78) {\footnotesize $\alpha$}; 
% Elektron
\shade[ball color=mycolor!25, opacity=1] (-6.63,3.1) circle (2pt);
   \node at (-6.63,3.1) {$-$};    
\shade[ball color=mycolor!25, opacity=1] (-4.0,2.64) circle (2pt);
    \node at (-4.0,2.64) {$-$};       
\end{scope}
%Achse z
\draw[->,>={Triangle[length=0pt 3*4,width=0pt 4]}, thick] (-5.40, 3.1) -- ++ (0.9, 0.45) -- ++ (-1.8, -0.9);
\end{scope}
\end{tikzpicture}
  \caption[Ablenkung eines Elektrons im E-Feld]{Ein Elektron wird durch eine Spannung $U_\mathrm{B}$ (nicht skizziert) in $x$-Richtung beschleunigt. Zwischen den Kondensatorplatten mit Länge $l$ erfährt das Elektron durch das E-Feld eine Beschleunigung parallel zur $x$-Achse, welche proportional zur Feldstärke $E$ ist. Ab $x=\sfrac{l}{2}$ bewegt sich die Ladung mit konstanter Geschwindigkeit $v= (\dot{x}, \dot{y})\tran$ weiter (eigene Darstellung).}
  \label{fig:efeld1}
  \vspace{-0pt}
\end{figure}