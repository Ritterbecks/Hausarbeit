  \thisfloatsetup{%
  capbesidewidth=\marginparwidth}

\begin{figure}[htb!p]
\centering
%\sansmath
\begin{tikzpicture}
\begin{axis}[	
	clip mode=individual, % Verhindet weiße Punkte bei vielfach geplotteten x-Werten
	ymajorgrids,
    xmajorgrids,
          axis x line*=middle,
          axis y line*=center,
    grid style={white,thick},
%	axis on top,
    width=12cm,
    height=8.797cm,
    xmin=-0.1,
    xmax=0.6,
    ymin=-0.025,
    ymax=0.05,
    %/tikz/ybar interval,
    tick align=center,
    xtick align=outside,
    xlabel={$x$},
    ylabel={$y$},   
   x label style={at={(axis cs:0.5,-0.010)}},
    y label style={at={(axis cs:-0.105,0.025)}},
    axis line style={Honeydew4!70!black},
    ticklabel style={Honeydew4!70!black, inner sep=1pt,
                font=\footnotesize},
    yticklabels={  -25, 0,  {$\SI{E-3}{\metre}$}, 50},            
    ytick={  -0.025, 0, 0.025, 0.050, 0.100},
    xtick={-0.1,0,0.1,0.2,0.3,0.4,0.5, 0.6}, 
    xticklabels={-0.1,0,0.1,0.2,0.3,0.4,{$\si{\metre}$}, 0.6},
    scaled ticks=false,
    width=\textwidth,                                   
    label style={font=\small,Honeydew4!70!black,},
    enlarge x limits=true,
    %tick style={draw=none},
    x tick label as interval=false,
       legend style ={ at={(axis cs:0.05,0.05)}, 
            anchor=north west, draw=none, 
            fill=none,align=left, text=Honeydew4!70!black, font=\footnotesize},
        cycle list name=mycolor4 white,
        smooth
    %nodes near coords={\pgfmathfloatifflags{\pgfplotspointmeta}{0}{}{\pgfmathprintnumber{\pgfplotspointmeta}}},
    %every node near coord/.append style={    fill=white,    anchor=mid west,        shift={(3pt,4pt)},    inner sep=0,    font=\footnotesize,    rotate=45},
]
\draw[thick, Honeydew4!70!black, ->,>={Kite[round, length=0.4cm, width=4pt]}] (axis cs:-0.105,0.015) -- (axis cs:-0.105,0.035);
\draw[thick, Honeydew4!70!black, ->,>={Kite[round, length=0.4cm, width=4pt]}] (axis cs:0.4,-0.010) -- (axis cs:0.6,-0.010);
\addplot
 table[x =x, y =y,]{images/data/5.dat};
 \addlegendentry{Bahn 1};
 \addplot
  table[x =x, y =y,]{images/data/6.dat};
  \addlegendentry{Bahn 2};
  \addplot
   table[x =x, y =y,]{images/data/7.dat};
   \addlegendentry{Bahn 3};
   \addplot
    table[x =x, y =y,]{images/data/8.dat};
    \addlegendentry{Bahn 4};
    \addplot
     table[x =x, y =y,]{images/data/9.dat};
     \addlegendentry{Bahn 5};
\end{axis}
\clipright
\end{tikzpicture}
  \caption[Transformierte Bewegung der ersten Kugel]{Transformierte Bewegung der ersten Kugel. Zu beachten ist, dass diese Kugel stärker von der Nulllinie abweicht als die Mehrzahl der übrigen 9 Kugeln.}
  \label{fig:trans2}
  \vspace{-0pt}
\end{figure}