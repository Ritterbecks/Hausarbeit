%% Autor: Björn Ritterbecks 
%% Letzte Aenderung: 15.06.2016 
\thisfloatsetup{%
  capbesidewidth=\marginparwidth}
\begin{figure}[htbp]
\centering
\usetikzlibrary{decorations.pathmorphing}
\pgfplotsset{width=7cm,compat=1.13}
\sffamily
\small
%\sansmath
\begin{tikzpicture}[
	scale=1,
	ka roehre/.style={fill=white,draw=black!80}
]
\begin{scope}[scale=1.14]

%Maßband
\draw[fill=mycolor5!50] (-1.00, 4.15) arc (35:-35:1.5 and 1.4) -- ++ (-0.2, -0.05) arc (-35:34:1.6 and 1.5) -- cycle;
\draw (-1.04, 3.95) -- (-0.86, 3.92);
\draw (-0.96, 3.75) -- (-0.79, 3.73);
\draw (-0.92, 3.55) -- (-0.75, 3.54);
\draw (-0.91, 3.35) -- (-0.73, 3.35);
\draw (-0.92, 3.15) -- (-0.74, 3.16);
\draw (-0.96, 2.95) -- (-0.77, 2.97);
\draw (-1.04, 2.75) -- (-0.87, 2.78);
%Verkabelung
\draw[thick] (-5.31,3.75) -- (-5.31,4.75);
\node[draw=none,fill=none] at (-5.31, 4.95){$+$};
\draw[thick] (-5.31,2.8) -- (-5.31,1.8);
\node[draw=none,fill=none] at (-5.31, 1.6){$-$};
\draw[thick] (-8.9, 3.35) -- (-10.1, 3.35);
\node[draw=none,fill=none] at (-10.0, 3.23){$-$};
\draw[thick] (-8.93, 4.75) -- (-8.93, 4.1) arc (180:270:0.4) -- ++ (0.6, 0);
\node[draw=none,fill=none] at (-8.93, 4.95){$+$};
%Kathode
\draw[fill=mycolor3!50] (-8.95,3.68) arc (90:270:0.08 and 0.34) -- ++ (0.1,0) arc (270:90:0.08 and 0.34) -- cycle;
\draw[fill=mycolor3!50] (-8.85,3.68) arc (90:-90:0.08 and 0.34)arc (270:90:0.08 and 0.34);
% Kollimatoren
\draw[fill=mycolor2!50] (-7.3,3.72) arc (90:270:0.1 and 0.38) -- ++ (0.2,0) arc (270:90:0.1 and 0.38) -- cycle;
\draw[fill=mycolor2!50] (-7.1,3.72) arc (90:-90:0.10 and 0.38)arc (270:90:0.1 and 0.38);
\draw[fill=mycolor2!50] (-8.1,3.72) arc (90:270:0.10 and 0.38) -- ++ (0.2,0) arc (270:90:0.1 and 0.38) -- cycle;
\draw[fill=mycolor2!50] (-7.9,3.72) arc (90:-90:0.1 and 0.38)arc (270:90:0.1 and 0.38);
\draw[fill=halfgray!20, opacity=0.7] (-7.18, 3.30) -- (-7.18, 3.32) -- (-7.02, 3.38) -- (-7.02, 3.36) -- cycle;
\draw[fill=halfgray!20, opacity=0.7] (-7.98, 3.28) -- (-7.98, 3.34) -- (-7.82, 3.40) -- (-7.82, 3.34) -- cycle;
% Drehwurm aka Vakuumröhre
\draw[mycolor!50] (-5.1,4) -- (-3.8,4) arc (270:315:0.5) arc (145:-145:1.5 and 1.4) arc (45:90:0.5) -- ++ (-1.3,0) arc (90:180:0.1) -- ++ (0, -0.15) arc (0:-45:0.2) arc (0:-180:0.05) arc (225:180:0.2) -- ++ (0, 0.15) arc (0:90:0.1) -- ++ (-1.3,0) arc (270:180:0.16) arc (0:90:0.1) -- ++ (-1.2,0)  arc (90:135:0.1) arc (315:180:0.7 and 0.4) arc (0:90:0.05) -- ++ (-0.2,0) arc (270:225:0.2) arc (270:90:0.05) arc (135:90:0.2) -- ++ (0.2,0) arc (270:360:0.05) arc (180:115:0.7 and 0.4) arc (270:360:0.1) -- ++ (0, 0.15) arc (180:135:0.2) arc (180:0:0.05) arc (45:0:0.2) -- ++ (0, -0.15) arc (180:270:0.1) arc (80:45:0.7 and 0.4) arc (225:270:0.1) -- ++ (1.2,0) arc (270:360:0.1) arc (180:90:0.16)-- ++ (1.3,0) arc (270:360:0.1) -- ++ (0, 0.15) arc (180:135:0.2) arc (180:0:0.05) arc (45:0:0.2) -- ++ (0, -0.15) arc (180:270:0.1) -- cycle; 
\shade[bottom color=mycolor!10, top color=mycolor!50,opacity=0.20] (-5.1,4) -- (-3.8,4) arc (270:315:0.5) arc (145:-145:1.5 and 1.4) arc (45:90:0.5) -- ++ (-1.3,0) arc (90:180:0.1) -- ++ (0, -0.15) arc (0:-45:0.2) arc (0:-180:0.05) arc (225:180:0.2) -- ++ (0, 0.15) arc (0:90:0.1) -- ++ (-1.3,0) arc (270:180:0.16) arc (0:90:0.1) -- ++ (-1.2,0)  arc (90:135:0.1) arc (315:180:0.7 and 0.4) arc (0:90:0.05) -- ++ (-0.2,0) arc (270:225:0.2) arc (270:90:0.05) arc (135:90:0.2) -- ++ (0.2,0) arc (270:360:0.05) arc (180:115:0.7 and 0.4) arc (270:360:0.1) -- ++ (0, 0.15) arc (180:135:0.2) arc (180:0:0.05) arc (45:0:0.2) -- ++ (0, -0.15) arc (180:270:0.1) arc (80:45:0.7 and 0.4) arc (225:270:0.1) -- ++ (1.2,0) arc (270:360:0.1) arc (180:90:0.16)-- ++ (1.3,0) arc (270:360:0.1) -- ++ (0, 0.15) arc (180:135:0.2) arc (180:0:0.05) arc (45:0:0.2) -- ++ (0, -0.15) arc (180:270:0.1) -- cycle; 
%B-Feld
\foreach \x in {0,...,3}
        { \foreach \y in {0,...,3} {
            \draw ({-5.8+0.3*\x},{3.8-0.3*\y}) node[mycolor2] {\textbf{.}};
            \draw[mycolor2] ({-5.8+0.3*\x},{3.8-0.3*\y}) circle (3pt);
        };      
};
%E-Feld
\draw[fill=mycolor!25] (-6.0, 2.8) -- (-5.8, 2.9) -- (-4.7, 2.9) -- (-4.9, 2.8);   \draw[fill=mycolor!25] (-6.0, 2.8) -- (-6.0, 2.75) -- (-4.9, 2.75) -- (-4.7, 2.85) -- (-4.7, 2.9) -- (-4.9, 2.8) -- (-6.0, 2.8);     
\draw (-4.9, 2.75) -- (-4.9, 2.8);
\draw[->,>={Triangle[length=0pt 3*3,width=0pt 3]}, thick, mycolor, shorten >=0.4pt] (-5.65,3.75) -- (-5.65,2.76) ;
\draw[->,>={Triangle[length=0pt 3*3,width=0pt 3]}, thick, mycolor, shorten >=0.4pt] (-5.35,3.75) -- (-5.35,2.76);
\draw[->,>={Triangle[length=0pt 3*3,width=0pt 3]}, thick, mycolor, shorten >=0.4pt] (-5.05,3.75) -- (-5.05,2.76);
\draw[fill=mycolor!75] (-6.0, 3.8) -- (-5.8, 3.9) -- (-4.7, 3.9) -- (-4.9, 3.8);
\draw[fill=mycolor!75] (-6.0, 3.8) -- (-6.0, 3.75) -- (-4.9, 3.75) -- (-4.7, 3.85) -- (-4.7, 3.9) -- (-4.9, 3.8) -- (-6.0, 3.8); 
\draw (-4.9, 3.75) -- (-4.9, 3.8);
\begin{scope}[thick, every node/.style={sloped,allow upside down}]
% Elektron
\draw[->,>={Triangle[length=0pt 3*3,width=0pt 3]},mycolor, thick] (-7.08, 3.35)  -- node {\midarrow} (-6.0, 3.35) arc (270:280:6.4)  -- node {\midarrow} ++ (3.8,0.66);
\draw[mycolor, thick] (-8.85, 3.35) -- node {\midarrow} (-8.2, 3.35); 
\draw[mycolor, thick] (-7.88, 3.35) -- node {\midarrow} (-7.40, 3.35);
\shade[ball color=mycolor!25, opacity=1] (-8.33,3.35) circle (3pt);
    \node at (-8.33,3.35) {$-$};
\shade[ball color=mycolor!25, opacity=1] (-6.23,3.35) circle (3pt);
    \node at (-6.23,3.35) {$-$};    
 \shade[ball color=mycolor!25, opacity=1] (-2.23,3.91) circle (3pt);
     \node at (-2.23,3.91) {$-$};       
\end{scope}
\end{scope}
\end{tikzpicture}
  \caption[Apparatur zur $\nicefrac{q}{m}$-Messung]{\textsc{Thomson}'sche Apparatur zur Messung von $\nicefrac{q}{m}$ aus dem Jahr 1897: Elektronen werden von der Kathode zu den Kollimatoren, deren Linker ebenfalls als Anode fungiert, beschleunigt, um anschließend in dem elektrischen und magnetischen Feld, die senkrecht aufeinander stehen, abgelenkt zu werden. Die Detektion erfolgt mit einer Maßskala am rechten Ende der Vakuumröhre (in Anlehnung an \cite[S.\,147]{Tipler2010}).}
  \label{fig:thoms1}
  \vspace{-0pt}
\end{figure}