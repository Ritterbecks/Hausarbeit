%% Autor: Björn Ritterbecks 
%% Letzte Aenderung: 15.06.2016 
\begin{marginfigure}
\centering
\begin{tikzpicture}
\begin{scope}[scale=0.5]

% Äquipotentiallinien  
\draw[dashed] (0,0) circle (1.0);    
\draw[dashed] (0,0) circle (1.8);    
\draw[dashed] (0,0) circle (2.8);    
 
%Kraft
\foreach \x in {1,...,24}
        {
                \draw[postaction={decorate},decoration={markings,mark=at position 0.62 with {\arrowreversed{Triangle[length=0pt 3*8,width=0pt 7]}}}, mycolor] 
                 (0,0) -- ({3.2*cos(\x*15)},{3.2*sin(\x*15)});
        }; 
 % Beschriftung
        \node[fill=white, rectangle] at (1.30,1.00) {\footnotesize $\boldsymbol{F}$};        
%Kraftrichtung        
 \draw[dotted, thick]  (0,0)   -- (2.86,2.19) ;         
% Ladungen                
 \shade[ball color=mycolor!25, opacity=1] (0.0,0) circle (10pt);
      \node at (0,0) {$-$}; 
      \draw[->,>={Triangle[length=0pt 3*4,width=0pt 4]},mycolor4, thick]  (2.86,2.19)   -- (1.59, 1.22) ;      
    \shade[ball color=mycolor!75, opacity=1] (2.86,2.19) circle (5pt);
         \node at (2.52,2.42) {\footnotesize $q_\mathrm{+}$};   
    
\end{scope}
\end{tikzpicture}
  \caption[Kugelsymmetrisches E-Feld]{Eine negative Ladung $Q$ erzeugt ein kugelsymmetrisches Feld (blau). Der Verlauf gleichen Potentials ist mit gestrichelten Linien angedeutet. Auf eine positive Elementarladung $q_\mathrm{+}$ wirkt eine Kraft $\boldsymbol{F}$ (nach \cite[S.\,6]{Demtroeder2009}).}
  \label{fig:feld2}
  \vspace{-0pt}
\end{marginfigure} 