\chapter{Fachdidaktik}
\label{kap:3}
\bookmarksetupnext{level=subsubsection}
\chapterinfo{Die Versuchsentwicklung erfolgt unter fachdidaktischen Überlegungen zur didaktischen Analayse,  der individuellen Wissenskonstruktion, der Festlegung eines Handlungsmusters zur Analogiebildung und den zu erreichenden Zieldimensionen.}

\textit{Die Versuchsentwicklung erfolgt unter fachdidaktischen Überlegungen zur didaktischen Analyse,  der individuellen Wissenskonstruktion, der Festlegung eines Handlungsmusters zur Analogiebildung und den zu erreichenden Zieldimensionen.}

\section{Wissenskonstruktion}

Nach \textsc{Christoph Lumer} kann \textit{Kognitivismus} als die These verstanden werden, dass nicht jede Problemstellung durch allgemeingültiges, lehrbares Wissen gelöst werden kann. Damit sollte es eigentlich \textit{Nonkognitivismus} heißen.\footfullcite[vgl.][S.\,695]{Lumer1999} Auf den Physikunterricht übertragen bedeutet dies, dass selbst bei perfektem (auswendiggelerntem) Wissen das Grundgerüst nicht vollends erschlossen sein muss. Eine wichtige Determinante der Kompetenz \textit{Fachwissen}, beziehungsweise der Fähigkeit, dieses anwenden zu können, ist die Verknüpfung mit Vorwissen und damit die Einbettung in eigene Wissensstrukturen

Aus dieser Warte betrachtet hat der Kognitivismus viele Gemeinsamkeiten mit dem \textit{Konstruktivismus}, der postuliert, dass Lernen ausnahmslos ein selbstgesteuerter Prozess ist. Es ist nach dessen Auffassung nicht möglich, Wissen zu übertragen. Lediglich ein Vokabelwissen kann antrainiert werden, jedoch nicht das problemlösende, heuristische Denken.

Wie \textsc{Dubs} es formuliert, ">bedarf es einer Interaktion zwischen dem Lerngegenstand und der lernenden Person"<,\footcite[S.\,29]{Dubs2009} was impliziert, dass der \textit{Nürnberger Trichter} bei der Wissensvermittlung wirkungslos ist. Es ist jedoch augenscheinlich, dass insbesondere in den sehr weit erforschten Gebieten der Naturwissenschaften nicht immer eine persönliche Interaktion mit dem zu erforschenden Gegenstand bzw. Begriff möglich ist. Fernerhin müsste der Lerner eine stark überdurchschnittliche Intelligenz mitbringen, um im Chemie- oder Physikunterricht mit ausschließlich konstruktivistischen Lehrmethoden einen gehobenen Wissensstandard zu erreichen.

Die Brücke zwischen Anspruch und Realität kann zumindest durch Experimentieren, reichhaltigen Medieneinsatz und Gedankenexperimente teilweise geschlagen werden. Hierzu bedarf es allerdings auch einer genauen Auseinandersetzung mit dem Vorwissen der Lernenden, da --- um es mit dem Konstruktivismus zu erklären --- Alltagserfahrungen, d.\,h. selbst Gefühltes oder Gesehenes zu gutem Behalten führen. Wenn demnach ein Schüler bereits eine Erklärung für ein Phänomen zur Hand hat, welche seiner Ansicht nach zufriedenstellend ist, so wird ein Umdenken und somit das Lernen der fachlich korrekten Zusammenhänge von vornherein erschwert oder sogar unmöglich gemacht. ">Der Unterricht muss also an die Vorstellungen der Schüler anknüpfen und ihre Eigenaktivitäten fördern."<\footfullcite[S.\,658]{Kircher2013}  

Aus diesen Gründen sind Analogexperimente so wichtig für das physikalische Studium und den Unterricht in der Schule: Mittels der Verbindung von Primär- und Sekundärbereich im Zusammenspiel mit eigenem Handeln fällt es sowohl leichter, Alltagsvorstellung zu überkommen, als auch das heuristische Denken zu trainieren. Überdies sind Analogien Bestandteil eines weiteren wirkungsmächtigen Zweig der Didaktik, der \textit{didaktischen Reduktion} nach \textsc{Wolfgang Klafki} (*1927).

\section{Didaktische Analyse}

Die Begriffe \textit{Elementarisierung} und \textit{didaktische Rekonstruktion} gehen auf \textsc{Klafki}s \textit{didaktische Analyse} mit ihren vier Leitfragen zurück:
\begin{items}
\item \textsw{Welchen} allgemeinen \textsw{Sinn} hat der Themenbereich? Hierbei sind mannigfaltige Differenzierungen möglich --- beispielsweise die gesellschaftliche Relevanz oder die Reflexion über das Thema --- womit ">\textit{der Gehalt eines Themas nicht eindeutig und nicht endgültig festgelegt} ist."<\footcite[S.\,78]{Kircher2013} Diese Zieldimension läuft auf einen stark interdisziplinären Unterricht hinaus, der als Lehrmittel bevorzugt exemplarisches Lernen einsetzt, wodurch sie in dieser Arbeit die am stärksten gewichtete Leitfrage sein wird.
\item \textsw{Welche Gegenwartsbedeutung} hat der Unterrichtsgegenstand für die Lerner? Um diese Frage beantworten zu können, bedarf es einer umfangreichen Lerngruppenanalyse, da außerschulische Faktoren --- insbesondere der \textit{soziokulturelle Hintergrund} --- stets eine Determinante für Lernerfolg sind. Werden die Lerner \textit{da abgeholt, wo sie stehen}, kann ein motivierender Einstieg gelingen. Bei der Massenspektrometrie wären beispielsweise die Lebensmittelanalyse oder --- besonders aktuell --- Dopingproben sinnstiftende Kontexte.\footfullcite{Mucke1995} 
\item Ähnlich wie der diejenige nach dem Gegenwartsbezug ist die Frage, \textsw{welche Zukunftsbedeutung} ein Thema hat. Gelernt werden soll damit nicht für die Schule, sondern für die Zukunft. Damit stehen auch die neuen \textit{Kulturtechniken} wie zum Beispiel das Verstehen und Erstellen von Diagrammen oder Nutzung von Messgeräten im Vordergrund.\footcite[vgl.][S.\,79]{Kircher2013} 
\item Die vierte und letzte Frage ist diejenige nach dem \textsw{Was}: Welche innere Struktur verfolgt der Unterrichtsgegenstand, wie kann er in die Begriffswelt der Physik eingeordnet werden, welches Fachwissen muss vermittelt werden, um das Thema zu erschließen?
\end{items}
Ausgehend von der didaktischen Analyse wird die Materie so in Glieder unterteilt, dass \textsc{alle Schüler} diese \textit{in möglichst kurzer Zeit gut} und \textit{auf humane Weise} erfassen können.\footcite[vgl.][S.\,108--137]{Kircher2013} An die Elementarisierung schließt ein Zusammensetzen der erarbeiteten \textit{Erklärungsglieder} an, wobei diese \textit{fachgerecht}, \textit{schülergerecht} und \textit{zielgerecht} zu einem Ganzen verknüpft werden sollen. Dies kann auf mehrere Arten erreicht werden, wobei für die Entwicklung eines Analogieversuches das \textit{gegenständliche Modell} in Kombination mit einer \textit{Analogiebildung} die zentralen sind.  

  \thisfloatsetup{
  capbesidewidth=\marginparwidth,}
\begin{table}[htbp]
\centering
%\sffamily,
\small
%\sansmath
\arrayrulecolor{white}
\vspace{0.2cm}
  \rowcolors{2}{halfgray!15}{halfgray!5}
 \setlength{\extrarowheight}{.4em}
			\begin{tabularx}{0.99\textwidth}{l*{1}{>{\RaggedRight\arraybackslash}X}}		
\rowcolor{mycolor}\multicolumn{1}{l}{{\color{white}\textbf{Schritt}}}&  \multicolumn{1}{l}{{\color{white}\textbf{Handlungsanweisungen}}}\\
1.: & Der Primärbereich $(O, M, E)$ in einer allgemeinen, auf das Vorwissen der Studierenden bezogenen Weise einführen.\\
2.: & Hinweise auf analoge, den Lernenden bekannte Lernbereiche $(O*, M*, E*)$ geben.\\
3.: & Isomorphismen von Primär- und Sekundärbereich herausfinden.\\
4.: & Listen über begriffliche und objektale Entsprechungen von $(O, M, E)$ in $(O*, M*, E*)$ anfertigen.\\
5.: & Hypothesenbildung zum Sekundärbereich, die experimentell überprüft werden sollen.\\
6.: & Eine Übertragung der Erkenntnisse auf den Primärbereich und das Überprüfen der Gesetze in $(O, M, E)$ sind in jedem Fall erforderlich.\\
7.: & Wo sind die Grenzen der Analogie, wo scheitert sie?\\
		\end{tabularx}
		\caption[Methodisches Muster der Analogiebildung]{Dargestellt ist das methodische Muster der Analogiebildung. $(O, M, E)$ beschreiben Objekte, Modelle und Experimente des Lernbereiches, $(O*, M*, E*)$ diejenigen des Analogiemodells. Der achte Schritt als metatheoretische Reflexion über Analogien ist nicht angegeben, da er für einen Praktikumsversuch im Studium nicht relevant ist (nach \cite[S.\,130]{Kircher2013}).} 
		\label{tab:ana1}
\vspace{0.2cm}		
		\end{table}  

\noindent Der Einsatz von Analogien lohnt genau dann, wenn der \textit{Trikolon} des oben genannten \textit{Gerechtwerdens} erfüllt wird. \footcite[vgl.][S.\,125]{Kircher2013} Faktoren, die einem Verstehen des primären Lernbereichs entgegenwirken können, sind unter anderem ein \textit{Akzeptanzproblem} bei geringer Ähnlichkeit der Analogie mit dem eigentlichen Lerngegenstand, \textit{irrelevante Merkmale} des Analogbereichs, die von den Lernenden einfach hingenommen werden müssen, und damit auch ein Hinterfragen der im Sekundärbereich aufgestellten \textit{Hypothese}. Um die Übertragbarkeit zwischen primärem und sekundärem Bereich zu befördern, folgt die Analogiebildung in Kapitel \ref{kap:4} dem methodischen Muster, das von \textsc{Kircher} et al. erarbeitet wurde und in Tabelle \ref{tab:ana1} wiedergegeben ist. Es erfolgt dabei für die Objekte $O$, Modelle $M$ und Experimente $E$ die Setzung, dass die Lernbereiche, die mit Asterisk dargestellt werden, denjenigen des Analogiebereiches entsprechen.

Bei einem Analogversuch zur Massenspektrometrie ist die experimentelle Überprüfung von \textit{Schritt 6} nur in den Unterbereichen möglich (vgl. z.\,B. das Fadenstrahlrohr, Kapitel \ref{kap:2}). Zudem werden die Schritte 3 und 4 nicht durch die Studenten ausgeführt, was die Wirksamkeit der Analogie weiter abschwächt.


\section{Zieldimensionen}

Es gilt für einen universitären Analogieversuch --- wie auch in der Schule --- der Grundsatz \textit{keine Aufgabe ohne Ziele}. Daher umreißt dieser Abschnitt den dreidimensionalen \textit{Lernzielraum}, welcher in Abbildung \ref{fig:ziele} zu sehen ist.
%% Autor: Björn Ritterbecks 
%% Letzte Aenderung: 15.06.2016 
\thisfloatsetup{%
  capbesidewidth=\marginparwidth,
}
\begin{figure}[htbp]
\centering
\usetikzlibrary{decorations.pathmorphing}
\pgfplotsset{width=7cm,compat=1.13}
\large
%\sansmath
\begin{tikzpicture}
\begin{scope}[scale=1.7]
\draw[->,>={latex[length=0pt 3*11,width=0pt 11]}, thick]  (0,0) -- ++ (0, 0.5) node [left=0.2cm] {Feinziel} -- ++ (0, 0.5) node [left=0.2cm] {Grobziel} -- ++ (0, 0.5) node [left=0.2cm] {Richtziel} -- ++ (0, 0.5) node [left=0.2cm] {Leitziel} -- ++ (0, 0.5) node [above] {\Large \textsc{Zielebene}};
\draw[->,>={latex[length=0pt 3*11,width=0pt 11]}, every node/.style={
    fill=white,
    anchor=north west,    
    shift={(-0.0cm,-0.2cm)},
    inner sep=0,
    font=\large,
    rotate=45}, thick]  (0,0) -- ++ (0.5, 0.0) node [below, left] {Reproduktion} -- ++ (0.5, 0.0) node [below, left] {Reorganisation} -- ++ (0.5, 0.0) node [below, left] {Transfer} -- ++ (0.5, 0.0) node [below, left] {Problemlösen}-- ++ (0.5, 0.0); 
\draw[every node/.style={
    fill=white,
    anchor=north west,    
    shift={(-1.4cm,-1.0cm)},
    inner sep=0,
    font=\large,
    rotate=45}, thick] (2.8,0) -- ++ (0.0,0.0) node {\Large \textsc{Zielstufe}};
\draw[->,>={latex[length=0pt 3*11,width=0pt 11]}, thick]  (0,0) -- ++ (-0.35, -.35) node [left=0.1cm]  {Konzeptziel} -- ++  (-0.35, -.35) node [left=0.1cm] {Prozessziel} -- ++  (-0.35, -.35) node [left=0.1cm]  {Soziales Ziel} -- ++  (-0.35, -.35) node [left=0.1cm]  {Einstellungen} -- ++  (-0.35, -.35) node [below] {\Large \textsc{Zielklasse}};
\shade[top color=mycolor!35, bottom color=mycolor!75, every node/.style={ midway}, thick, opacity=0.3] (-0.7, -0.7) -- ++ (1.5, 0) -- ++ (0, 1.5) -- ++ (-1.5, 0) -- ++ (0, -1.5) -- ++ (0.7, 0.7) -- ++ (1.5, 0) -- ++ (0, 1.5) -- ++ (-1.5, 0) -- ++ (-0.7, -0.7) -- ++ (1.5,0) -- ++ (0.7, 0.7) -- ++(0, -1.5) -- ++ (-0.7, -0.7) -- ++ (0, 1.5) -- ++ (0.7, 0.7) -- cycle; 
\shade[top color=mycolor!35, bottom color=mycolor!75, every node/.style={ midway}, thick, opacity=0.3] (1.5, 0) -- ++ (0, 1.5) -- ++ (-0.7, -0.7) -- ++ (0, -1.5) -- cycle; 
\shade[top color=mycolor!55, bottom color=mycolor!75, every node/.style={ midway}, thick, opacity=0.3] (-0.7, -0.7) -- ++ (1.5, 0) -- ++ (0, 0.7) -- ++ (-0.8, 0) -- cycle; 
\draw[every node/.style={ midway}] (-0.7, -0.7) -- ++ (1.5, 0) -- ++ (0, 1.5) -- ++ (-1.5, 0) -- ++ (0, -1.5) -- ++ (0.7, 0.7) -- ++ (1.5, 0) -- ++ (0, 1.5) -- ++ (-1.5, 0) -- ++ (-0.7, -0.7) -- ++ (1.5,0) -- ++ (0.7, 0.7) -- ++(0, -1.5) -- ++ (-0.7, -0.7) -- ++ (0, 1.5) -- ++ (0.7, 0.7);   
\draw[->,>={latex[length=0pt 3*11,width=0pt 11]}, thick, mycolor4]  (0,0) -- ++ (0.8, 0.8) node [above, right] {Lernziel $\boldsymbol{z}_\mathrm{kes}$};
\foreach \x in {1,...,4}
        { \draw ({-0.05}, {0+0.5*\x}) -- ++ ({0.1}, {0.0});
        };
\foreach \x in {1,...,4}
        { \draw ({-0.35*\x-0.05*cos(45)}, {-0.35*\x+0.05*sin(45)}) -- ++ ({0.1*cos(45)},{-0.1*sin(45)}) ;
        };  
\foreach \x in {1,...,4}
        { \draw ({0+0.5*\x},{-0.05} ) -- ++ ({0.0}, {0.1});
        };              
\end{scope} 
\end{tikzpicture}
\caption[Der Lernzielraum]{Darstellung des dreidimensionalen Lernzielraumes. Als Beispiel wird das Richtziel \textsw{Verständnis der Massenspektrometrie} dargestellt, was Grobziele und Feinziele der Zielebene, Konzept- und Prozessziele als Zielklasse und die Reproduktion, Reorganisation und den Transfer, d.\,h. die dritte Zielstufe mit einschließt. Es ist nicht zwingend erforderlich, dass das gesamte Volumen unter dem Lernziel $\boldsymbol{z}_\mathrm{kes}$ mitgedacht wird. Es gibt vor allem in der Teilchenphysik der Oberstufe Ziele, die nur Einstellungen und den Transfer als Leitziele verfolgen (nach \cite[S.\,89]{Kircher2013}).}
  \label{fig:ziele}
\end{figure}

Es erfolgt eine Unterteilung in vier \textit{Zieldimensionen}, deren Stufen nicht immer streng hierarchisch zu sehen sind:
\begin{items}
\item Die \textsw{Zielebenen} nach \textsc{Westphalen} (1979) besitzen noch immer eine Relevanz für die Lehrerausbildung \footcite[vgl.][S.\,84]{Kircher2013}. Sie werden unterteilt in \textit{Leitziele}, \textit{Richtziele}, \textit{Grobziele} und  \textit{Feinziele}, die \textsc{Kircher} folgendermaßen zusammenfasst:
\begin{quote}
\textit{Ein Leitziel} kann als Motto \textit{über dem Eingang eines Schulhauses} angebracht werden. \textit{Ein Richtziel} kann über der \textit{Tür zum Physikraum} stehen. \textit{Ein Grobziel} kann als \textit{Stundenthema an die Tafel} geschrieben werden. \textit{Feinziele} sind im Physikheft als \textit{Merksätze} [...].\footcite[S.\,87]{Kircher2013}
\end{quote}
\item \textsw{Zielklassen} sind diejenige Zieldimension, welche in den vier Kompetenzbereichen aus dem \textit{Kernlehrplan Physik (G8)} Nordrhein-Westfalens eine Entsprechung findet.

Hierbei sind \textit{Konzeptziele} die \textit{kognitiven} Ziele wie unter anderem \textit{das Wissen über} physikalische Phänomene, Begriffe, Prinzipien und Fakten.

Die \textit{Prozessziele} umfassen sämtliche Fertigkeiten im Bereich der \textit{Untersuchungsmethoden}. \textsc{Klopfer} (1971) differenziert hierbei fünf \textit{Anforderungsbereiche}\footcite[vgl.][S.\,90]{Kircher2013}, die streng hierarchisch sind. Der unterste Anforderungsbereich ist beispielsweise das \textit{Messen} oder \textit{Beobachten}, der höchste umschreibt die Fähigkeit zur \textit{methodologischen Reflexion}.

\textit{Soziale Ziele} verfolgen \textit{Kommunikations-} und \textit{Kooperationsbereitschaft} u.\,v.\,m.: Sie sind demnach die \textit{soft skills}, welche immer wieder im Kontext der Anforderungen des modernen Arbeitsmarktes genannt werden.

Bei den \textit{Zielen über Einstellungen und Werte} wird ein verantwortungsvoller Umgang mit Mitmenschen, der Natur und Technik angestrebt.
\item Die \textsw{Zielstufen} entsprechen den Anforderungsbereichen des Kernlehrplans, wobei nach \textit{Reproduktion} und \textit{Anwendung} der dritte Bereich noch einmal in \textit{Transfer} und \textit{Problemlösung} unterteilt ist.
\end{items}
Diese Hausarbeit wird \textsc{Kircher}s Vorgaben nicht vollständig folgen, da z.\,B. eine Differenzierung zwischen \textit{Problemlösen} und \textit{Transfer} zu weit ginge. Eine Übertragung von Gelerntem auf ein neues \textit{Problem} denkt stets auch die Problemlösung mit, da der Transfer andernfalls ohnehin nicht notwendig wäre. 

Ebenfalls werden soziale Ziele und Einstellungen eher als eine einzige Zielklasse gesehen, da das Leitmotiv des \textit{mündigen Bürgers} ein Sammelbegriff dieser Beiden ist (entspricht der \textit{Bewertung} im Kernlehrplan), jedoch als vierte Zielklasse die \textit{kommunikativen Kompetenzen}, wie Diagrammerstellung, deren Interpretation, Präsentationsstärke und das Nachvollziehen selbiger, ergänzt.  
Damit sind die Zieldimensionen und -bereiche darstellbar mit Tabelle \ref{tab:ziele1}.
\vspace*{-3.2cm}
  \thisfloatsetup{
  capbesidewidth=\marginparwidth,}
\begin{margintable}
\centering
%\sffamily,
\small
%\sansmath
\arrayrulecolor{white}
\vspace{0.2cm}
  \rowcolors{2}{halfgray!15}{halfgray!5}
 \setlength{\extrarowheight}{.0em}
			\begin{tabularx}{0.99\marginparwidth}{l*{1}{>{\RaggedRight\arraybackslash}X}}		
\rowcolor{mycolor}\multicolumn{1}{l}{{\color{white}\textbf{Dim.}}}&  \multicolumn{1}{l}{{\color{white}\textbf{Niveau}}}\\
 & Feinziel\\
\rowcolor{halfgray!5}& Grobziel\\
 & Richtziel\\
\rowcolor{halfgray!5}\multirow{-4}{*}{\textsc{Ebene}} & Leitziel\\[4pt]
\rowcolor{halfgray!15}& Reproduktion\\
 & Reorganisation\\
\rowcolor{halfgray!15}\multirow{-3}{*}{\textsc{Stufe}} & Problemlösen\\[4pt]
\rowcolor{halfgray!5} & Konzeptziel\\
& Prozessziel\\
\rowcolor{halfgray!5} & Komm.\\
\multirow{-4}{*}{\textsc{Klasse}} & Bewertung\\
		\end{tabularx}
		\caption[Ziele]{Darstellung der Zieldimensionen und -stufen, wie sie für den Analogieversuch verwendet werden (in Anlehnung an \cite[S.\,76--92]{Kircher2013}).} 
		\label{tab:ziele1}		
		\end{margintable}