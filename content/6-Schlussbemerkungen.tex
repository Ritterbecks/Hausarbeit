\chapter{Schlussbemerkungen}
\label{kap:6}
\bookmarksetupnext{level=paragraph}
\chapterinfo{Mittels der experimentell gewonnenen Daten wird ein mögliches Grobkonzept für einen Praktikumsversuch nebst möglicher Lernziele vorgestellt, worauf ein persönliches Fazit dieser Examensarbeit folgt.}

\textit{Mittels der experimentell gewonnenen Daten wird ein mögliches Grobkonzept für einen Praktikumsversuch nebst möglicher Lernziele vorgestellt, worauf ein persönliches Fazit dieser Examensarbeit folgt}

\section{Versuchskonzept}

Um einen motivierenden Analogieversuch durchführen zu können, wäre es vorteilhaft, alle Kugelarten gleich einzufärben, damit sie nur haptisch unterschieden werden können. Bei einer Benutzung von beispielsweise unlackierten Aluminium- und Acrylkugeln könnte bei den Studierenden ein Akzeptanzproblem der Analogie auftreten, da man im Regelfall bei der Massenspektrometrie vor der Untersuchung auch nicht weiß, wie der Ausgang sein wird.

Werden vier Kugelradien und -arten genutzt, ergibt sich die Möglichkeit, mindestens zwei Dichte-Radius-Kombinationen einzusetzen, die das gleiche Masse-Projektionsfläche-Verhältnis (bzw. im Primärbereich das gleiche $\sfrac{q}{m}$-Verhältnis) aufweisen. Dieses ungleiche Paar kann als Einstieg --- nachdem die Kugeln herumgereicht worden sind --- analysiert werden. Das spürbar unterscheidbare Gewicht in Verbindung mit der gleichen Aufspaltung führt zu einem \textit{kognitiven Konflikt}, der zu den wichtigsten Hilfsmitteln für motivierende Einstiege und langfristiges Behalten zählt. 

Dabei soll darauf geachtet werden, die Kugeln von der gleichen Höhe abzurollen, da anschließend zwei nicht von den Studierenden ertastete Kugeln gleichen Gewichts heruntergerollt werden, woran eine kurze Diskussion anschließen kann. 

Die eigentliche Durchführung des Versuches bietet mannigfaltige Herangehensweisen. Es wäre selbst eine Videoanalyse durchführbar, die jedoch stark reduziert sein müsste: Im Praktikumsversuch \textsc{VID} des I. Physikalischen Instituts der RWTH Aachen wird lediglich ein einziges Video aufgezeichnet und ausgewertet, jedoch füllt der Versuch nichtsdestotrotz einen halben Nachmittag. Im Vordergrund muss stets die Analyse von Probensubstanzen stehen. Ob einzelne \textit{Elemente}, oder eine durchmischte Probe untersucht werden, unterscheidet sich allerdings nur im Schwierigkeitsgrad. 

Auf Grundlage eines Lernzielkatalogs kann eine genaue Passung der Versuchsanleitung und -aufgaben erfolgen. Differenziert nach Studiengängen kann das Anforderungsniveau flexibel angepasst werden. Im Bereich der Zielebenen werden folgende Lernziele vorgeschlagen:
\begin{items}
\item Wie bei den meisten Gruppenversuchen können als \textsw{Leitziele} das heu{\-}ris{\-}tisch-problemlösende Denken und die Fähigkeit zur konstruktiven Gruppenarbeit gesehen werden. Insbesondere das Problemlösen als höchstes Anforderungsniveau kann durch einen Analogieversuch zu einem geschwindigkeitsfokussierenden Massenspektrometer geschult werden. Wird lediglich die Näherungsfunktion des Strömungsfeldes nach Kapitel \ref{kap:5} bei der Forderung nach einer vollständigen Analyse der $\sfrac{q}{m}$-Verhältnisse inklusive Betrachtung der Messungenauigkeiten angegeben, könnten jedoch auch einige Studenten schnell demotiviert werden und nicht weiter am Versuchsgeschehen teilnehmen.
\item Anknüpfend an die Leitziele kann das Übertragen des Analogmodells auf den Lernbereich \textit{Massenspektrometrie} als gehobenes \textsw{Richtziel} gesehen werden. Da die Massenanalyse so viele physikalische Gesetze vereint und kombiniert, stellt ihr Verständnis ebenfalls ein Richtziel dar. Weiterhin gehören auch das experimentelle Arbeiten und Modellieren von Zusammenhängen in diesen Zielbereich. 
\item \textsc{Kircher} betont, dass \textsw{Feinziele} lediglich in der Lehrerausbildung einen begrifflichen Wert haben, da die Operatoren wie \textit{Verstehen} ohne Lerngruppenanalyse nicht auf ihre Erfüllung hin beurteilt werden können.\footcite[vgl.][S\,86]{Kircher2013}
\item Die \textsw{Grobziele}, welche bei der Auseinandersetzung mit dem hier beschriebenen Funktionsmodell verfolgt werden können, sind vielfältig. Neben den Gravitationskraft und den Strömungsfeldern des Analogiebereiches muss die Theorie der Wirkungsweise von elektrischen und magnetischen Feldern wiederholt werden. Die Verbindung dieser Bereiche über Isomorphie-Betrachtungen wie beispielsweise
\begin{equation*}
qU_\mathrm{B}\cong mgh
\end{equation*}
führt zu einer Vernetzung von Wissen und damit zu einem erleichterten Übergang in das Langzeitgedächtnis.
\end{items}
Die Erweiterung des Funktionsmodell um einen Geschwindigkeitsfilter bringt zwar Schwierigkeiten in der Kalibration mit sich, bietet jedoch großes Lernpotential und verschiebt den Versuch auf der mathematisch-naturwissenschaftlichen Skala etwas weiter in Richtung der \textit{Physik}.\vspace*{-1.5cm}\footcite[vgl.][, dessen Konzeption aufgrund der Limitationen des vorigen Analogiemodells eher auf die \textsw{Stochastik} ausgerichtet war. (S.\,48--49)]{Mais2014}\vspace*{1.5cm}

\section{Fazit}

Da die Massenspektrometrie eine in seiner Bedeutung herausstechende Analysemethode für die (Sport-)medizin, Naturwissenschaften und Industrie  darstellt, wurde im Zuge dieser Hausarbeit der Versuch unternommen, ein geschwindigkeitsfokussierendes Funktionsmodell zu entwickeln, das auch abseits von Demonstrationsexperimenten und qualitativen Schülerversuchen eingesetzt werden kann. Physikstudenten sollte zwar die Möglichkeit eingeräumt werden, an einem aktualem Massenspektrometer experimentieren zu dürfen, jedoch ist es sowohl aus Gründen des Fachwissens als auch der Interessen für Nebenfachstudierende sinnvoller --- zumindest zu Beginn --- das Funktionsprinzip an einem Analogiemodell zu erarbeiten.

Die hier vorgeschlagene Analogie erfüllt die beiden Kriterien nach \textsc{Kircher}: Der Modellversuch und \textsc{Aston}s Massenspektrometer sind in der Bewegungsbeschreibung der Teilchen \textit{oberflächenähnlich}, die Begriffe und Objekte von Primär- und Sekundärbereich sind weitestgehend \textit{isomorph}. Die Unzulänglichkeiten von Analogiemodellen sollen in einem Praktikumsversuch hingegen nicht erörtert werden.\footcite[vgl.][S.\,128--129]{Kircher2013}  

Es konnte jedem Objekt des primären Lernbereichs im Analogieversuch begrifflich-mathematisch eine Entsprechung zugewiesen werden. Insbesondere die Energieerhaltung, Bewegungsgleichungen im Coulomb-Feld, die Impulsänderung beim Passieren eines Magnetfeldes, Masse und Ladung finden ihre Entsprechungen im entwickelten Versuchsaufbau.

Die ausgewählten Ähnlichkeitsbeziehungen konnten experimentell bestätigt werden, obwohl die Ungenauigkeit der selbstgebauten Startrampe und besonders der Holz- und Glaskugeln dazu beigetragen haben, dass eine große Menge an Messdaten durch Fehlmessungen immer weiter abgeschmolzen wurde. 

Bei einer professionell gefertigten Ausführung liegt ein enormes Wertschöpfungspotential in dem vorgestellten Funktionsmodell, das --- meines Wissens --- das Erste ist, welches Impuls- und Energieanalyse, Massen- und Ladungsanalogien miteinander vereint. Diese Menge an Parametern, die bildlich als Stellschrauben gesehen werden können, führte dazu, dass ständig Komponenten verändert werden mussten. Obwohl ich um das frühe Stadium des Prototypen weiß, bin ich überzeugt, dass der Versuch \textsc{MAS} in naher Zukunft seinen Platz in den physikalischen Praktika für Nebenfachstudierende finden wird.



