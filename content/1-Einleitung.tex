%\RaggedRight
\chapter{Einleitung}
\bookmarksetupnext{level=3}
\chapterinfo{Es existieren einige wenige Analogieexperimente zur Massenspektrometrie, welche eine qualitative Betrachtung ermöglichen. Diese Arbeit unternimmt den Versuch, eine quantifizierbare Analogie zu entwickeln.}

   % \marginpar{
   % \includegraphics[width=\marginparwidth]{images/saeulendiagramm}
   % \captionof{figure}{Text of the caption}
   % }
   
\textit{Es existieren einige wenige Analogieexperimente zur Massenspektrometrie, welche eine qualitative Betrachtung ermöglichen. Diese Arbeit unternimmt den Versuch, eine quantifizierbare Analogie zu entwickeln.}

\section{Motivation}
\label{kap:1}
Der Begriff \textit{Massenspektrometer} (bzw. \textit{Massenfilter} oder \textit{Massenspektrograph}) beschreibt ">ein Instrument, das aus einer Substanzprobe [...] gasförmige Ionen erzeugt, diese nach Masse und Ladung trennt und schließlich ein \textit{Massenspektrum} (MS) liefert."<\vspace*{-0.9cm}\footfullcite[S.\,7]{Budz2005}\vspace*{0.9cm} Die Massenspektrometrie geht auf \textsc{J.\,J.\,Thomson}s Kathodenstrahlexperimente im Jahre 1897 zurück und kann --- historisch gesehen --- als Nebenprodukt der Erforschung der Atomstruktur betrachtet werden.\par
Nach \textsc{J.\,H.\,Gross} ist das Verständnis der Massenspektrometrie in den Naturwissenschaften Biologie, Chemie und Physik, in den Ingenieurswissenschaften, der Medizin und Pharmazie für Studierende\vspace*{-0.9cm}\footnote{In dieser Arbeit wird aus Gründen des Leseflusses im Folgenden stets der männliche Genus verwendet.}\vspace*{0.9cm} und Praktizierende von imperativer Bedeutung.\footfullcite[vgl.][S.\,1]{Gross2012} Dies äußert sich ebenfalls in einer von der RWTH Aachen im Jahr 2012 durchgeführten Umfrage, in welcher die Lehrkörper von Hochschulen mit dem Studiengang Biologie nach den wichtigsten Lehrinhalten für angehende Biologen befragt worden sind.\footfullcite[vgl.][S.\,1]{Boehmer2013} Aus diesen Gründen besteht ein Bedarf für Experimente zur Massenspektrometrie in physikalischen Praktika auf Hochschulebene, den die vorliegende Arbeit stillen will.\par
In den gängigen universitären Praktika gibt es mannigfaltige Ausprägungen von Experimenten. So werden Sachzusammenhänge unter anderem direkt, als Simulationen oder Analogieexperimente erfahrbar gemacht, was in Abhängigkeit vom spezifischen Versuchsaufbau über das \textit{Tun}, die \textit{Haptik}, eine \textit{Elementarisierung} oder die \textit{Vernetzung mit Bekanntem} das Lernen und Behalten fördert --- oder in einigen Fällen --- überhaupt erst ermöglicht.\par
Wie persönliche Erfahrungen zeigen, steigt zudem der Lernerfolg mit der Komplexität der Anforderungen, da durch intensives Recherchieren und Nachdenken die Konstruktion von Wissensrepräsentationen befördert wird. So ist es durchaus möglich, ein Analogieexperiment vollständig und richtig auszuwerten, ohne es auf den Primärbereich übertragen zu können. Jedoch gerade die Vernetzung von bereits Gelerntem, bestenfalls sogar Erfahrenem mit neuen Konzepten führt zu langfristigem und tiefgreifendem Lernen, wie es uns sowohl der (gemässigte) Konstruktivismus\footfullcite[vgl.][S.\,27--31]{Dubs2009} als auch der Kognitivismus aufzeigen. Möglichst intensive Verwebungen von Wissensgebieten führen dazu, dass Analogien ein hervorragendes Lehr- und Lernmittel darstellen, falls die Aufgabenstellungen und Erläuterungen auf eine direkte Übertragbarkeit gerichtet werden und diese zur Auswertung benötigt wird. \par Der Einsatz eines modernen Massenspektrographen kommt für Studierende mit Nebenfach Physik nicht in Frage, da die Massenspektrometrie ein klassischer \textit{Blackbox}-Versuch ist und die Komplexität und der Anspruch mit moderneren Geräten steigen, was \textsc{Gross} mit einem Globus paraphrasiert, bei dem man stets nur einen Anteil der gesamten Oberfläche sehen kann, jedoch nie das Ganze.\footcite[vgl.][S.\,6]{Gross2012}  Überdies kann ein Analogieversuch auch, falls eine weitere didaktische Reduktion erfolgt, als Demonstrationsexperiment im Physikunterricht der Oberstufe durchgeführt werden, was diese Versuchsform als die logische Wahl erscheinen lässt.

\section{Überblick}
Diese Hausarbeit befasst sich daher mit der Entwicklung eines Analogieversuchs zur Massenspektrometrie, der neben einem intuitiven Zugang vielfältige Möglichkeiten zur Ausarbeitung auf unterschiedlichen Anforderungsniveaus bieten soll.\par
Im Vorfeld dieses Textes wurden zwei Staatsexamensarbeiten am \textsc{I.\,Physikalischen Institut IA} mit einem gleichlautendem Themengebiet verfasst, auf deren zentrale Ergebnisse an entsprechenden Stellen eingegangen wird,\footfullcite[vgl.][und \textsc{Böhmer, 2013}]{Mais2014} die allerdings aufgrund der Veränderung jedes einzelnen Bestandteiles der Analogie nur als Ansatzpunkte einfließen.\par 
Hierzu folgt auf die theoretischen Grundlagen der Massenspektrometrie in Kapitel \ref{kap:2} --- mit besonderer Berücksichtigung des Massenspektrometers von \textsc{Aston} --- der fachdidaktische Orientierungsrahmen (Kapitel \ref{kap:3}) für eine dezidierte Analogiebetrachtung im \ref{kap:4}. Kapitel, bei der ebenfalls die Bestandteile des Analogieversuches theoretisch begründet und praktisch entwickelt werden.\par
Im Hauptteil wird, nach der Überprüfung der Analogiekomponenten hinsichtlich ihrer Übertragbarkeit von Theorie auf Praxis und einer mathematischen Modellbildung, der Versuchsaufbau finalisiert und eine Probemessung ausgewertet (Kapitel \ref{kap:5}).\par
Die Schlussbemerkungen enden mit einem Fazit, das an Überlegungen und Konzepte für eine Versuchsanleitung anknüpft.



