  \thisfloatsetup{
  capbesidewidth=\marginparwidth,
}
\begin{table}[htb]
\centering
%\sffamily,
\small
%\sansmath
\arrayrulecolor{white}
%\setlength{\arrayrulewidth}{2pt}
\vspace{0.2cm}
  \rowcolors{2}{halfgray!15}{halfgray!5}
 \setlength{\extrarowheight}{.00em}
			\begin{tabularx}{0.99\textwidth}{*{2}{>{\RaggedLeft\arraybackslash}X}r*{2}{>{\RaggedLeft\arraybackslash}X}}		
\rowcolor{mycolor}  
%\multicolumn{1}{c}{\color{white}\textbf{Radius $\boldsymbol{r}$}} &
\multicolumn{1}{c}{\color{white}\textbf{Abstand $\boldsymbol{s}$}} &  \multicolumn{1}{c}{\color{white}\textbf{ $\boldsymbol{10^{-3}\cdot R\kern-.04em e}$}} &  \multicolumn{1}{c}{\color{white}\textbf{$\boldsymbol{c_\mathrm{W,\, exp}}$}} &  \multicolumn{1}{c}{\color{white}\textbf{$\boldsymbol{c_\mathrm{W,\, theo}}$}} &  \multicolumn{1}{c}{\color{white}\textbf{Abweichung}}\\ \rowcolor{mycolor}
% \multicolumn{1}{c}{\color{white}\textbf{in $\boldsymbol{\si{\milli\metre}}$}} &
  \multicolumn{1}{c}{\color{white}\textbf{in $\boldsymbol{\si{\centi\metre}}$}} &  \multicolumn{1}{c}{\color{white}\textbf{o.\,E.}} &  \multicolumn{1}{c}{\color{white}\textbf{o.\,E.}} &  \multicolumn{1}{c}{\color{white}\textbf{o.\,E.}} &  \multicolumn{1}{c}{\color{white}\textbf{in $\si{\percent}$}}\\
1,0	$\pm$	0,1	&	78,46	&	0,49	&	0,50	&	-0,4	\\
5,0	$\pm$	0,2	&	72,32	&	0,50	&	0,49	&	1,3	\\
10,0	$\pm$	0,3	&	63,45	&	0,46	&	0,49	&	-5,1	\\
15,0	$\pm$	0,5	&	55,95	&	0,49	&	0,48	&	2,6	\\
20,0	$\pm$	0,7	&	49,81	&	0,50	&	0,47	&	5,3	\\
25,0	$\pm$	0,9	&	44,35	&	0,50	&	0,47	&	7,0	\\
30,0	$\pm$	1,1	&	30,70	&	0,47	&	0,46	&	2,1	\\
		\end{tabularx}
		\caption[Gegenüberstellung experimenteller und theoretischer $c_\mathrm{W}$-Werte]{Gegenüberstellung experimenteller und theoretischer $c_\mathrm{W}$-Werte für Stufe 2 und $\SI{17.5}{\milli\metre}$ Radius. Bei den Geschwindigkeiten, die das Gebläse erreicht, ist keine Korrektur für kompressible Fluide nötig. Es zeigt sich, dass die Messungen gut mit den erwartbaren Werten übereinstimmen und eine Näherung mit $c_\mathrm{W}= 0,45\pm 0,05$ ausreichend gewesen wäre.}
		\label{tab:rey}	
		\end{table} %\vspace*{-5cm}\newpage