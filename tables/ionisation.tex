  \thisfloatsetup{
  capbesidewidth=\marginparwidth,}
\begin{table}[htbp]
\centering
%\sffamily,
\small
%\sansmath
\arrayrulecolor{white}
\vspace{0.2cm}
  \rowcolors{2}{halfgray!15}{halfgray!5}
 \setlength{\extrarowheight}{.4em}
			\begin{tabularx}{0.99\textwidth}{l*{1}{>{\RaggedRight\arraybackslash}X}}		
\rowcolor{mycolor}\multicolumn{1}{l}{{\color{white}\textbf{Primärbereich}}}&  \multicolumn{1}{l}{{\color{white}\textbf{Analogie}}}\\
Elektronenstoßquelle & Pistole\\
Elektronen mit kinetischer Energie $\approx \SI{70}{\electronvolt}$ & abgefeuerte Munition\\
Moleküle/Atome & Glasscheibe \\
Ionen/Fragmente usw. & Scherben unterschiedlicher Größe\\
		\end{tabularx}
		\caption[Analogon zu der Ionisationskammer]{Analogiebetrachtungen der Ionisationskammer eines Massenspektrometers. Wird wiederholt mit einer Pistole auf Glasscheiben gefeuert, ergeben sich randomisierte Scherbengrößen-Verteilungen, die als unterschiedliche Ionenausbeuten betrachtet werden können, wenn einem Größenintervall das Prädikat ">ionisiertes Molekül"< zugeordnet wird (nach \cite[S.\,91]{Blaum2011}).} 
		\label{tab:ionisation}
\vspace{0.2cm}		
		\end{table}