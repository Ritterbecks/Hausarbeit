  \thisfloatsetup{
  capbesidewidth=\marginparwidth,}
\begin{table}[htb]
\centering
%\sffamily,
\small
%\sansmath
\arrayrulecolor{white}
\vspace{0.2cm}
  \rowcolors{2}{halfgray!15}{halfgray!5}
 \setlength{\extrarowheight}{.0em}
			\begin{tabularx}{0.99\textwidth}{l*{1}{>{\RaggedRight\arraybackslash}X}}		
\rowcolor{mycolor}\multicolumn{1}{l}{{\color{white}\textbf{Sektorfeld-Massenspektrometer}}}&  \multicolumn{1}{l}{{\color{white}\textbf{Modellversuch}}}\\
Atome & Acrylkugeln\\
Ionen & mit Eisenwolle gefüllte Kugeln\\
Beschleunigereinheit & Schiefe Ebene\\
Massenanalysator  & Permanentmagnet\\
homogenes Magnetfeld & inhomogenes Magnetfeld\\
Detektor & Röhrensystem\\\bottomrule[5pt]
Elektrisches Feld  & Gravitationsfeld\\
Lorentzkraft $F_\mathrm{L}$& Magnetische Kraft $F_\mathrm{m}$ (sic!)\\
Ionenladung $q$ & Volumen $V_\mathrm{EW}$ der Eisenwolle\\
Elementarladung $e$  & Feste Volumeneinheit $V=\frac{\sfrac{1}{3}\si{\gram}}{\rho_\mathrm{EW}}$\\
Ladung $q=n\cdot e$  & $n\cdot V$\\
		\end{tabularx}
		\caption[Analogien Modellversuch Magnete]{Die Analogie-Zuweisungen nach \textsc{Böhmer} und \textsc{Mais} für den Analogieversuch mit einer Magnetanordnung. Im oberen Teil der Tabelle sind die objektalen, in der unteren Hälfte die begrifflichen Zuordnungen aufgelistet (nach \cite[S.\,16]{Mais2014}).} 
		\label{tab:mais}		
		\end{table}