  \thisfloatsetup{
  capbesidewidth=\marginparwidth,
}
\begin{table}[htb]
\centering
%\sffamily,
\small
%\sansmath
\arrayrulecolor{white}
%\setlength{\arrayrulewidth}{2pt}
\vspace{0.2cm}
  \rowcolors{2}{halfgray!15}{halfgray!5}
 \setlength{\extrarowheight}{.00em}
 \subfloat[$s=\SI{1}{\centi\metre}$]{
			\begin{tabularx}{0.45\textwidth}{r*{2}{>{\RaggedLeft\arraybackslash}X}}	
\rowcolor{mycolor} & 	\multicolumn{2}{c}{{\color{white}\textbf{Kraft in $\boldsymbol{\si{\milli\newton}}$}}} \\
\rowcolor{mycolor}  \multirow{-2}{*}{{\color{white}\textbf{$\boldsymbol{d}$ in $\boldsymbol{\si{\centi\metre}}$}}} & \multicolumn{1}{c}{{\color{white}\textbf{Stufe 2}}} &\multicolumn{1}{c}{{\color{white}\textbf{Stufe 1}}}\\
-4,0	$\pm$	0,1	&	0	&	0	\\
-3,5	$\pm$	0,1	&	6	&	2	\\
-3,0	$\pm$	0,1	&	60	&	42	\\
-2,5	$\pm$	0,1	&	115	&	70	\\
-2,0	$\pm$	0,1	&	150	&	90	\\
-1,5	$\pm$	0,1	&	210	&	122	\\
-1,0	$\pm$	0,1	&	276	&	177	\\
-0,5	$\pm$	0,1	&	303	&	188	\\
0,0	$\pm$	0,1	&	292	&	183	\\
0,5	$\pm$	0,1	&	263	&	165	\\
1,0	$\pm$	0,1	&	214	&	128	\\
1,5	$\pm$	0,1	&	154	&	97	\\
2,0	$\pm$	0,1	&	114	&	68	\\
2,5	$\pm$	0,1	&	75	&	52	\\
3,0	$\pm$	0,1	&	23	&	14	\\
3,5	$\pm$	0,1	&	0	&	0	\\
		\end{tabularx}}
		\quad
 \subfloat[$s=\SI{5}{\centi\metre}$]{
			\begin{tabularx}{0.45\textwidth}{r*{2}{>{\RaggedLeft\arraybackslash}X}}	
\rowcolor{mycolor} & 	\multicolumn{2}{c}{{\color{white}\textbf{Kraft in $\boldsymbol{\si{\milli\newton}}$}}} \\
\rowcolor{mycolor}  \multirow{-2}{*}{{\color{white}\textbf{$\boldsymbol{d}$ in $\boldsymbol{\si{\centi\metre}}$}}} & \multicolumn{1}{c}{{\color{white}\textbf{Stufe 2}}} &\multicolumn{1}{c}{{\color{white}\textbf{Stufe 1}}}\\
-4,0	$\pm$	0,2	&	0	&	0	\\
-3,5	$\pm$	0,2	&	3	&	0	\\
-3,0	$\pm$	0,2	&	32	&	5	\\
-2,5	$\pm$	0,2	&	67	&	35	\\
-2,0	$\pm$	0,2	&	111	&	71	\\
-1,5	$\pm$	0,2	&	170	&	104	\\
-1,0	$\pm$	0,2	&	217	&	145	\\
-0,5	$\pm$	0,2	&	249	&	156	\\
0,0	$\pm$	0,2	&	250	&	157	\\
0,5	$\pm$	0,2	&	235	&	146	\\
1,0	$\pm$	0,2	&	205	&	121	\\
1,5	$\pm$	0,2	&	159	&	84	\\
2,0	$\pm$	0,2	&	114	&	62	\\
2,5	$\pm$	0,2	&	65	&	30	\\
3,0	$\pm$	0,2	&	10	&	2	\\
3,5	$\pm$	0,2	&	0	&	0	\\
		\end{tabularx}}		
		\caption[Exemplarische Kraftmessung des Gebläses]{Exemplarische Kraftmessung des Gebläses für eine Kugel mit $r=\SI{17.5}{\milli\metre}$ bei einem Abstand von \textbf{\color{mycolor}(a)}, $s=\SI{1}{\centi\metre}$ und \textbf{\color{mycolor}(b)}, $s=\SI{5}{\centi\metre}$ vom Föhn, der über zwei Gebläsestufen verfügt. Die Aufnahme der Messdaten erfolgt mit einem \textsw{Coulomb-Kraftmesser}.} 
		\label{tab:kraftmessung1}	
		\end{table} %\vspace*{-5cm}\newpage