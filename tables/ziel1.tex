  \thisfloatsetup{
  capbesidewidth=\marginparwidth,}
\begin{margintable}
\centering
%\sffamily,
\small
%\sansmath
\arrayrulecolor{white}
\vspace{0.2cm}
  \rowcolors{2}{halfgray!15}{halfgray!5}
 \setlength{\extrarowheight}{.0em}
			\begin{tabularx}{0.99\marginparwidth}{l*{1}{>{\RaggedRight\arraybackslash}X}}		
\rowcolor{mycolor}\multicolumn{1}{l}{{\color{white}\textbf{Dim.}}}&  \multicolumn{1}{l}{{\color{white}\textbf{Niveau}}}\\
 & Feinziel\\
\rowcolor{halfgray!5}& Grobziel\\
 & Richtziel\\
\rowcolor{halfgray!5}\multirow{-4}{*}{\textsc{Ebene}} & Leitziel\\[4pt]
\rowcolor{halfgray!15}& Reproduktion\\
 & Reorganisation\\
\rowcolor{halfgray!15}\multirow{-3}{*}{\textsc{Stufe}} & Problemlösen\\[4pt]
\rowcolor{halfgray!5} & Konzeptziel\\
& Prozessziel\\
\rowcolor{halfgray!5} & Komm.\\
\multirow{-4}{*}{\textsc{Klasse}} & Bewertung\\
		\end{tabularx}
		\caption[Ziele]{Darstellung der Zieldimensionen und -stufen, wie sie für den Analogieversuch verwendet werden (in Anlehnung an \cite[S.\,76--92]{Kircher2013}).} 
		\label{tab:ziele1}		
		\end{margintable}