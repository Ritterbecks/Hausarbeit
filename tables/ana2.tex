  \thisfloatsetup{
  capbesidewidth=\marginparwidth,}
\begin{table}[htb]
\centering
%\sffamily,
\small
%\sansmath
\arrayrulecolor{white}
\vspace{0.2cm}
  \rowcolors{2}{halfgray!15}{halfgray!5}
 \setlength{\extrarowheight}{.0em}
			\begin{tabularx}{0.99\textwidth}{l*{1}{>{\RaggedRight\arraybackslash}X}}		
\rowcolor{mycolor}\multicolumn{1}{l}{{\color{white}\textbf{Sektorfeld-Massenspektrometer}}}&  \multicolumn{1}{l}{{\color{white}\textbf{Funktionsmodell}}}\\
verschiedene Atome & Kugeln unterscheidbarer Dichte\\
ionisierte Atome & Verschiedene Kugelquerschnittsflächen\\
Hochvakuum & Leerer Experimentiertisch\\
Beschleunigungsspannung & Startrampe mit waagerechtem Auslauf\\
Blendensystem & Schmale Platte lässt nicht alle Geschwindigkeiten zu\\
Analysator I  & Schiefe Ebene\\
Analysator II  & Haartrockner\\
Photoplatte & Auffangbehälter\\
		\end{tabularx}
		\caption[Objektebene der Analogie]{Die Objektebene des Analogieversuches mit einer schiefen Ebene und einem Haartrockner als Massenanalysatoren (eigene Darstellung).} 
		\label{tab:ana2}		
		\end{table}