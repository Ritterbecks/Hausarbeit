  \thisfloatsetup{
  capbesidewidth=\marginparwidth,
}
\begin{table}[htb]
\centering
%\sffamily,
\small
%\sansmath
\arrayrulecolor{white}
%\setlength{\arrayrulewidth}{2pt}
\vspace{0.2cm}
  \rowcolors{2}{halfgray!15}{halfgray!5}
 \setlength{\extrarowheight}{.00em}
			\begin{tabularx}{0.99\textwidth}{c*{1}{>{\RaggedLeft\arraybackslash}X}r*{2}{>{\RaggedLeft\arraybackslash}X}}		
\rowcolor{mycolor}  
%\multicolumn{1}{c}{\color{white}\textbf{Radius $\boldsymbol{r}$}} &
\multicolumn{1}{c}{\color{white}\textbf{Kugel}} &  \multicolumn{1}{c}{\color{white}\textbf{$\boldsymbol{E_\mathrm{pot}}$ in}} &  \multicolumn{1}{c}{\color{white}\textbf{$\boldsymbol{E_\mathrm{rot}}$ in}} &  \multicolumn{1}{c}{\color{white}\textbf{$\boldsymbol{E_\mathrm{trans}}$ in}} &  \multicolumn{1}{c}{\color{white}\textbf{Abweichung}}\\ \rowcolor{mycolor}
% \multicolumn{1}{c}{\color{white}\textbf{in $\boldsymbol{\si{\milli\metre}}$}} &
  \multicolumn{1}{c}{\color{white}\textbf{Nr.}} &    \multicolumn{1}{c}{\color{white}\textbf{$\boldsymbol{\si{\milli\newton\metre}}$}}  &    \multicolumn{1}{c}{\color{white}\textbf{$\boldsymbol{\si{\milli\newton\metre}}$}}  &    \multicolumn{1}{c}{\color{white}\textbf{$\boldsymbol{\si{\milli\newton\metre}}$}}  &  \multicolumn{1}{c}{\color{white}\textbf{in $\si{\percent}$}}\\
1	&		&	0,873	&	2,135	&	-22,5	\\
2	&		&	0,821	&	1,983	&	-27,8	\\
3	&		&	0,896	&	2,162	&	-21,3	\\
4	&		&	0,799	&	2,058	&	-26,4	\\
5	&		&	0,846	&	2,047	&	-25,5	\\
6	&		&	0,978	&	2,258	&	-16,7	\\
7	&		&	0,870	&	2,248	&	-19,7	\\
8	&		&	0,952	&	2,220	&	-18,3	\\
9	&	\multirow{-9}{*}{3,884}	&	0,819	&	2,216	&	-21,8	\\
		\end{tabularx}
		\caption[Umsetzung der potentiellen in kinetische Energie]{Videoanalyse einer Messreihe zur Nullachsenbestimmung. Es werden die Gesamtgeschwindigkeiten der ersten $\SI{0.2}{\second}$ zur Berechnung der Translations- und Rotationsenergie verwendet. Die Summe beider --- die kinetische Energie --- wird als prozentuale Abweichung zur Lageenergie ausgedrückt.}
		\label{tab:energy}	
		\end{table} %\vspace*{-5cm}\newpage