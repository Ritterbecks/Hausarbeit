  \thisfloatsetup{
  capbesidewidth=\marginparwidth,}
\begin{table}[htbp]
\centering
%\sffamily,
\small
%\sansmath
\arrayrulecolor{white}
\vspace{0.2cm}
  \rowcolors{2}{halfgray!15}{halfgray!5}
 \setlength{\extrarowheight}{.4em}
			\begin{tabularx}{0.99\textwidth}{l*{1}{>{\RaggedRight\arraybackslash}X}}		
\rowcolor{mycolor}\multicolumn{1}{l}{{\color{white}\textbf{Schritt}}}&  \multicolumn{1}{l}{{\color{white}\textbf{Handlungsanweisungen}}}\\
1.: & Der Primärbereich $(O, M, E)$ in einer allgemeinen, auf das Vorwissen der Studierenden bezogenen Weise einführen.\\
2.: & Hinweise auf analoge, den Lernenden bekannte Lernbereiche $(O*, M*, E*)$ geben.\\
3.: & Isomorphismen von Primär- und Sekundärbereich herausfinden.\\
4.: & Listen über begriffliche und objektale Entsprechungen von $(O, M, E)$ in $(O*, M*, E*)$ anfertigen.\\
5.: & Hypothesenbildung zum Sekundärbereich, die experimentell überprüft werden sollen.\\
6.: & Eine Übertragung der Erkenntnisse auf den Primärbereich und das Überprüfen der Gesetze in $(O, M, E)$ sind in jedem Fall erforderlich.\\
7.: & Wo sind die Grenzen der Analogie, wo scheitert sie?\\
		\end{tabularx}
		\caption[Methodisches Muster der Analogiebildung]{Dargestellt ist das methodische Muster der Analogiebildung. $(O, M, E)$ beschreiben Objekte, Modelle und Experimente des Lernbereiches, $(O*, M*, E*)$ diejenigen des Analogiemodells. Der achte Schritt als metatheoretische Reflexion über Analogien ist nicht angegeben, da er für einen Praktikumsversuch im Studium nicht relevant ist (nach \cite[S.\,130]{Kircher2013}).} 
		\label{tab:ana1}
\vspace{0.2cm}		
		\end{table}