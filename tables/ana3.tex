  \thisfloatsetup{
  capbesidewidth=\marginparwidth,}
\begin{table}[htb]
\centering
%\sffamily,
\small
%\sansmath
\arrayrulecolor{white}
\vspace{0.2cm}
  \rowcolors{2}{halfgray!15}{halfgray!5}
 \setlength{\extrarowheight}{.40em}
			\begin{tabularx}{0.99\textwidth}{X*{1}{>{\RaggedRight\arraybackslash}X}}		
\rowcolor{mycolor}\multicolumn{1}{l}{{\color{white}\textbf{Sektorfeld-Massenspektrometer}}}&  \multicolumn{1}{l}{{\color{white}\textbf{Funktionsmodell}}}\\
Coulomb-Potential $\phi_\mathrm{Coul} =\frac{1}{4\uppi\epsilon_0}\frac{Q}{r}$ & Gravitationspotential $\phi_\mathrm{Grav}=-G\frac{m}{r}$\\
Elementarladung $q$ & Projektionsfläche $A_\mathrm{min}=\uppi r_\mathrm{min}^2$\\
Gesamtladung $Q=n\cdot q$ & Projektionsfläche $A_i=i\cdot A_\mathrm{min}$\\ 
Potentielle Energie $qU_\mathrm{B}$& Potentielle Energie $mgh$\\
Kinetische Energie $\frac{1}{2}mv^2$& Kinetische Energie $\frac{7}{10}mv^2$\\
Lorentzkraft $F_\mathrm{L}=qvB$ & Strömungswiderstandskraft $F_\mathrm{W}=c_\mathrm{W}A\frac{\rho v^2}{2}$\\
Anzahl detektierter Impulse für $\sfrac{m}{q}=j$ & Aufgefangene Kugeln in Detektor $j$\\
		\end{tabularx}
		\caption[Begriffsebene der Analogie]{Übersicht der begrifflichen Analogien zwischen dem Primär- und Sekundärbereich bei Nutzung einer schiefen Ebene und eines Föhns als Massenanalysatoren. Da bei dem Vergleich zwischen den Lernbereichen der Radius $r$ auftaucht, wird ab dieser Stelle für die Geschwindigkeit $v$ verwendet (eigene Darstellung).} 
		\label{tab:ana3}		
		\end{table}